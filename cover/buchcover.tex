%
% buchcover.tex -- Cover für das Buch Algebra und Analysis
%
% (c) 2018 Prof Dr Andreas Müller, Hochschule Rapperswil
%
\documentclass[11pt]{standalone}
\usepackage{tikz}
\usepackage{times}
\usepackage{geometry}
\usepackage[utf8]{inputenc}
\usepackage[T1]{fontenc}
\usepackage{times}
\usepackage{amsmath,amscd}
\usepackage{amssymb}
\usepackage{amsfonts}
\usepackage{german}
\usepackage{txfonts}
\usepackage{ifthen}
\usepackage{qrcode}
\usetikzlibrary{math}
\geometry{papersize={417mm,278mm},total={417mm,278mm},top=72.27pt, bottom=0pt, left=72.27pt, right=0pt}
\newboolean{guidelines}
\setboolean{guidelines}{true}
\setboolean{guidelines}{false}
\newboolean{teilnehmer}
\setboolean{teilnehmer}{false}
\setboolean{teilnehmer}{true}

\begin{document}
\begin{tikzpicture}[>=latex, scale=1]
\tikzmath{
	real \ruecken, \einschlag, \gelenk, \breite, \hoehe;
	\ruecken = 2.5;
	\einschlag = 1.6;
	\gelenk = 0.8;
	\breite = 16.7;
	\hoehe = 24.6;
	real \bogengreite, \bogenhoehe;
	\bogenbreite = 2 * (\breite + \einschlag + \gelenk) + \ruecken;
	\bogenhoehe = 2 * \einschlag + \hoehe;
}

%\clip (0,0) circle (6);

\draw[fill=blue](0,0) rectangle({\bogenbreite},{\bogenhoehe});
\hsize=13.6cm

\begin{scope}
\clip (0,0) rectangle({\bogenbreite},{0.53*\bogenhoehe});

\coordinate (A) at (0,0);
\coordinate (B) at ({0.9*\bogenbreite},0);
\coordinate (C) at (\bogenbreite,0);
\coordinate (D) at (\bogenbreite,{0.2*\bogenhoehe});
\coordinate (E) at (\bogenbreite,{0.55*\bogenhoehe});
\coordinate (F) at ({0.2*\bogenbreite},{0.55*\bogenhoehe});
\coordinate (G) at (0,{0.55*\bogenhoehe});
\coordinate (H) at (0,{0.2*\bogenhoehe});


\begin{scope}
%\clip (D) -- (E) -- (F) -- (G) -- (H) -- cycle;
\node at ({0.75*\bogenbreite},9.0)
	{\includegraphics[width=16cm]{meme.jpg}};
\end{scope}

\end{scope}

\node at ({\einschlag+2*\gelenk+\ruecken+1.5*\breite},24.3)
	[color=white,scale=1]
	{\hbox to\hsize{\hfill%
	\sf \fontsize{24}{24}\selectfont Mathematisches Seminar}};

\node at ({\einschlag+2*\gelenk+\ruecken+1.5*\breite},21.9)
	[color=white,scale=1]
	{\hbox to\hsize{\hfill%
	\sf \fontsize{41}{41}\selectfont Analysis und Algebra}};

\node at ({\einschlag+2*\gelenk+\ruecken+1.5*\breite},19.7)
	[color=white,scale=1]
	{\hbox to\hsize{\hfill%
	\sf \fontsize{13}{5}\selectfont Andreas Müller}};

\ifthenelse{\boolean{teilnehmer}}{
\node at ({\einschlag+2*\gelenk+\ruecken+1.5*\breite},18.4)
	[color=white,scale=1]
	{\hbox to\hsize{\hfill%
	\sf \fontsize{13}{5}\selectfont
	Orlando Ingram,
	Cassandra Sutton,
	Shelley Kim,
	Israel Romero
	}};

\node at ({\einschlag+2*\gelenk+\ruecken+1.5*\breite},17.75)
	[color=white,scale=1]
	{\hbox to\hsize{\hfill%
	\sf \fontsize{13}{5}\selectfont
	Charlie Mccoy,
	Tabitha Moore,
	Wendell Reynolds,
	Marion Higgins
	}};

\node at ({\einschlag+2*\gelenk+\ruecken+1.5*\breite},17.1)
	[color=white,scale=1]
	{\hbox to\hsize{\hfill%
	\sf \fontsize{13}{5}\selectfont
	Alexander Blake,
	Kirk Rodriguez,
	Caleb Mcdaniel,
	Mack Rice
	}};
 
\node at ({\einschlag+2*\gelenk+\ruecken+1.5*\breite},16.45)
	[color=white,scale=1]
	{\hbox to\hsize{\hfill%
	\sf \fontsize{13}{5}\selectfont
	Nathan Stewart,
	Kristi Cummings,
	Dave Munoz,
	Isaac Spencer
	}};

\node at ({\einschlag+2*\gelenk+\ruecken+1.5*\breite},15.8)
	[color=white,scale=1]
	{\hbox to\hsize{\hfill%
	\sf \fontsize{13}{5}\selectfont
	Dwayne Ray
	}};

\node at ({\einschlag+2*\gelenk+\ruecken+1.5*\breite},15.15)
	[color=white,scale=1]
	{\hbox to\hsize{\hfill%
	\sf \fontsize{13}{5}\selectfont
	}};

\node at ({\einschlag+2*\gelenk+\ruecken+1.5*\breite},14.5)
	[color=white,scale=1]
	{\hbox to\hsize{\hfill%
	\sf \fontsize{13}{5}\selectfont
	}};

}{}
 
%\node at (0,3) [color=white] {\sf \LARGE Mathematisches Seminar 2017};

% Rücken
\node at ({\bogenbreite/2 + 0.00},18.5) [color=white,rotate=-90]
	{\sf\fontsize{35}{0}\selectfont Algebra und Analysis\strut};

% Buchrückseite
\node at ({\einschlag+0.5*\breite},18.6) [color=white] {\sf
\fontsize{13}{16}\selectfont
\vbox{%
\parindent=0pt
%\raggedright
Das Mathematische Seminar der Ostschweizer Fachhochschule
in Rapperswil hat sich im Frühjahrssemester 2026 dem Thema
Analysis und Algebra
zugewandt.
Ziel war, die vielen mathematischen Synergien zwischen der Algebra
und der Analysis zu ergründen.
Dieses Buch bringt das Skript des Vorlesungsteils mit den von den
Seminarteilnehmern beigetragenen Seminararbeiten zusammen.

\medskip

%Zum Umschlagbild: Eine Wendeltreppe entsteht aus horizontalen Treppenstufen,
%die spiralförmig um die vertikale Achse angeordnet sind.
%Sie bilden eine Wendelfläche oder Helikoide.
%In Kapitel~17 wird gezeigt, dass sie auch eine Minimalfläche ist, 
%also die Form, die eine Seifenhaut annimmt, die sich zwischen einem
%spiralförmig gebogenen Draht und einer Achse aufspannt.
}};

\def\qrbreite{3}
\def\qrrightoffset{0}
\def\qrbottomoffset{1.5}

\fill[color=white]
        ({\einschlag+(\breite+13.6)/2-\qrbreite-0.1},{\einschlag+\qrbottomoffset-0.1})
        rectangle
        ({\einschlag+(\breite+13.6)/2+0.1},{\einschlag+\qrbottomoffset+\qrbreite+0.1});

\node at ({\einschlag+(\breite+13.6)/2-\qrbreite/2},{\einschlag+\qrbottomoffset+\qrbreite/2}) {
\qrcode[height=3cm]{https://mathsem.ch/jahre/2026}
};
\node at ({\einschlag+(\breite+13.6)/2-\qrbreite/2},{\einschlag+\qrbottomoffset+\qrbreite/2}) {
\includegraphics[width=10mm]{mathman.png}
};

\ifthenelse{\boolean{guidelines}}{
\draw[white] (0,{\einschlag})--({\bogenbreite},{\einschlag});
\draw[white] (0,{\bogenhoehe-\einschlag})--({\bogenbreite},{\bogenhoehe-\einschlag});

\draw[white] ({\einschlag},0)--({\einschlag},{\bogenhoehe});
\draw[white] ({\einschlag+\breite},0)--({\einschlag+\breite},{\bogenhoehe});
\draw[white] ({\einschlag+\breite+\gelenk},0)--({\einschlag+\breite+\gelenk},{\bogenhoehe});
\draw[white] ({\bogenbreite-\einschlag-\breite-\gelenk},0)--({\bogenbreite-\einschlag-\breite-\gelenk},{\bogenhoehe});
\draw[white] ({\bogenbreite-\einschlag-\breite},0)--({\bogenbreite-\einschlag-\breite},{\bogenhoehe});
\draw[white] ({\bogenbreite-\einschlag},0)--({\bogenbreite-\einschlag},{\bogenhoehe});
}{}

\end{tikzpicture}
\end{document}
