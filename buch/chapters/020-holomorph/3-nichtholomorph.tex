%
% 3-nichtholomorph.tex
%
% (c) 2025 Prof Dr Andreas Müller
%
\section{Nicht holomorphe Funktionen
\label{buch:holomorph:section:nichtholomorph}}
\kopfrechts{Nicht holomorphe Funktionen}
Die Cauchy-Riemann-Differentialgleichungen zeigen, dass nicht jede
Funktion, die als Funktion $\mathbb{R}^2\to\mathbb{R}^2$ betrachtet
differenzierbar ist, auch holomorph ist.
In diesem Abschnitt soll eine weitere Charakterisierung holomorpher
Funktionen gezeigt werden.

%
% Funktionen von $z$ und $\bar{z}$
%
\subsection{Funktionen von $z$ und $\bar{z}$}
Bisher haben wir zur Berechnung der Ableitung einer komplexen Funktion
$f(z)$ als Funktionen der reellen Variablen $x$ und $y$ mit $z=x+iy$
betrachtet.
Die Funktion $z\mapsto \operatorname{Re}z=x$ und
$z\mapsto \operatorname{Im}z=y$ können aber als reellwertige Funktionen
nicht holomorph sein.
Die gleiche Information, die in $x$ und $y$ steckt, lässt sich aber auch
aus $z$ und $\bar{z}$ herausholen.
Da $z\mapsto z$ ganz offensichtlich eine holomorphe Funktion ist, scheint
dies eine vorteilhaftere Parametrisierung zu sein.

%
% z und \bar{z} als unabhängige Variablen
%
\subsubsection{$z$ und $\bar{z}$ als unabhängige Variablen}
Die Potenzfunktionen $z\mapsto z^k$ und alle Funktionen, die sich als 
Potenzreihen in $z$ ausdrücken lassen, konnten dank der Rechenregeln für
die komplexe Ableitung sehr leicht als holomorph erkannt werden.
Für Funktionen, die nur durch Realteil $u(x,y)$ und Imaginärteil $v(x,y)$
von Real- und Imaginärteil von $z$ ausgedrückt sind, mussten die
Cauchy-Riemann-Differentialgleichungen beigezogen werden, um Holomorphie
zu entscheiden.
Wir versuchen diese Vorgehensweise dadurch zu vereinfachen, dass wir
$z$ und $\bar{z}$ als unabhängige Variablen verwenden.

Für die Umrechnung zwischen $x$-$y$-Parametern und $z$-$\bar{z}$-Parametern
verwenden wir die Beziehungen
\begin{equation}
\left.
\begin{aligned}
     z  &= x + iy \\
\bar{z} &= x - iy 
\end{aligned}
\right\}
\qquad\Leftrightarrow\qquad
\left\{
\begin{aligned}
x &= \frac12    (z + \bar{z}) \\
y &= \frac1{2i} (z - \bar{z})
\end{aligned}
\right.
\label{buch:holomorph:nichtholomorph:eqn:xyzzbar}
\end{equation}

\begin{beispiel}
Wir betrachten die Funktion $f(z) = u(x,y)+iv(x,y)$ mit
$u(x,y)=x^2-y^2$ und $v(x,y)=2xy$.
Ersetzen wir mit Hilfe der Formeln
\eqref{buch:holomorph:nichtholomorph:eqn:xyzzbar}
die Variablen $x$ und $y$ durch $z$ und $\bar{z}$, dann ist
\begin{align*}
u(x,y)
&=
\frac14(z^2 + 2z\bar{z} + \bar{z}^2)
-
\frac1{-4}(z^2 - 2z\bar{z} + \bar{z}^2)
=
\frac14(2z^2 + 2\bar{z}^2)
\\
v(x,y)
&=
2
\frac1{4i}(z + \bar{z})(z - \bar{z})
=
\frac{1}{2i}(z^2 - \bar{z}^2)
\\
f(z)
&=
\frac12(z^2+\bar{z}^2)
+
\frac12(z^2-\bar{z}^2)
=
z^2.
\end{align*}
Die Rechnung hat also gezeigt, dass $f(z)$ sich auch allein als eine
Funktion von $z$ ausdrücken lässt, die Variable $\bar{z}$ wird nicht
gebraucht, um einen algebraischen Ausdruck für $f(z)$ zu konstruieren.
\end{beispiel}

\begin{beispiel}
\label{buch:holomorph:nichtholomorph:bsp:bbetrag2}
Wenden wir
\eqref{buch:holomorph:nichtholomorph:eqn:xyzzbar}
auf die Funktion $f(z) = x^2 + y^2$ 
an, erhalten wir
\begin{align*}
f(z)
&=
\frac14(z+\bar{z})^2
+
\frac1{-4}(z-\bar{z})^2
=
\frac14\bigl(
z^2 + 2z\bar{z} + \bar{z}^2
-
(z^2 - 2z\bar{z} + \bar{z}^2)
\bigr)
=
z\bar{z}.
\end{align*}
Die Funktion $f(z)=x^2+y^2=|z|^2$ lässt sich also nicht allein mit
$z$ ausdrücken.
\end{beispiel}

Wenn der Imaginärteil der Funktion $f(z)$ verschwindet, müsste nach den
Cauchy-Riemann-Differentialgleichungen 
\[
\frac{\partial u}{\partial x} = 0
\qquad\text{und}\qquad
\frac{\partial u}{\partial y} = 0
\]
sein, somit dürfte $u$ weder von $x$ noch von $y$ abhängen, $f(z)$ müsste
eine reelle Konstante sein, wenn Sie holomorph sein sollte.
Die Funktion von Beispiel~\ref{buch:holomorph:nichtholomorph:bsp:bbetrag2}
kann also nicht holomorph sein.

%
% Ableitungen nach $z$ und $\bar{z}$
%
\subsubsection{Ableitungen nach $z$ und $\bar{z}$}
Verschwindet eine partielle Ableitung einer Funktion von mehreren
reellen Variablen, dann bedeutet dies, dass die Funktion nicht
von der betreffenden Variable abhängt.
Es muss also einen Ausdruck geben, in dem diese Variable gar nicht
vorkommt.
Die Beispiele des vorangegangenen Abschnittes haben angedeutet, dass 
eine holomorphe Funktion nicht von $\bar{z}$ sondern nur von $z$ abhängt.
Es stellt sich daher die Frage, ob man diese Eigenschaft auch durch das
Verschwinden einer Art partieller Ableitung nach $\bar{z}$ ausdrücken
kann.

In
Abschnitt~\ref{buch:holomoprh:holomorph:subsection:komplexedifferenzierbarkeit}
wurde gezeigt, dass die komplexe Ableitung sowohl durch Ableiten
nach dem Realteil wie auch nach dem Imaginärteil gefunden werden
kann.
Es wurde gefunden, dass
\[
\frac{df}{dz}
=
\frac{\partial f}{\partial x}
=
\frac{1}{i}
\frac{\partial f}{\partial y}
=
-i\frac{\partial f}{\partial y}
\]
ist.
Zwar lässt sich die komplexe Ableitung mit diesen Formeln oft sehr
leicht berechnen.
Aus mathematischer Sicht sind sie jedoch nicht sehr schön, da sie den
Real- bzw.~den Imaginärteil bevorzugen.
Dies kann durch Bildung des Mittelwertes vermieden werden, es ist nämlich
\begin{equation}
\frac{df}{dz}
=
\frac{1}{2}
\biggl(
\frac{\partial f}{\partial x}
-i
\frac{\partial f}{\partial y}
\biggr).
\label{buch:holomorph:nichtholomorph:eqn:partialddz}
\end{equation}
Setzt man die Zerlegung $f=u+iv$ in Real- und Imaginärteil ein, kann
man mit den Cauchy-Riemann-Differentialgleichungen die ursprünglichen
Formeln zurückgewinnen:
\begin{align*}
\frac12
\biggl(
\frac{\partial f}{\partial x}
-i
\frac{\partial f}{\partial y}
\biggr)
&=
\frac12
\biggl(
\frac{\partial u}{\partial x}+i\frac{\partial v}{\partial x}
-
i
\biggl(
\frac{\partial v}{\partial y} + i\frac{\partial v}{\partial y}
\biggr)
\biggr)
=
\frac12
\biggl(
\frac{\partial u}{\partial x}
+
\frac{\partial v}{\partial y}
\biggr)
+
\frac{i}2
\biggl(
\frac{\partial v}{\partial x} - \frac{\partial u}{\partial y}
\biggr)
\\
&=
\begin{cases}
\displaystyle
\frac12
\biggl(
\frac{\partial u}{\partial x}
+
{\color{darkred}\frac{\partial u}{\partial x}}
\biggr)
+
\frac{i}2
\biggl(
\frac{\partial v}{\partial x}
+
{\color{darkred}\frac{\partial v}{\partial x}}
\biggr)
=
\frac{\partial f}{\partial x}
&\quad
\text{aus
$\displaystyle
\frac{\partial v}{\partial y}
=
{\color{darkred}
\frac{\partial u}{\partial x}}$
und
$\displaystyle
\frac{\partial u}{\partial y}
=
{\color{darkred}-\frac{\partial v}{\partial x}}$,}
\\[10pt]
\displaystyle
\frac12\biggl(
{\color{darkred}\frac{\partial v}{\partial y}}
+
\frac{\partial v}{\partial y}
\biggr)
+
\frac{i}2\biggl(
{\color{darkred}-\frac{\partial u}{\partial y}}
-
\frac{\partial u}{\partial y}
\biggr)
=
\frac{1}{i}
\frac{\partial f}{\partial y}
&\quad
\text{aus
$\displaystyle
\frac{\partial u}{\partial x}
=
{\color{darkred}\frac{\partial v}{\partial y}}
$
und
$\displaystyle
\frac{\partial v}{\partial x}
=
{\color{darkred}-\frac{\partial u}{\partial y}}
$.}
\end{cases}
\end{align*}
Die Cauchy-Riemann-Differentialgleichungen zeigen also, dass der 
``partielle'' Ableitungsoperator $\partial/\partial z$ mit dem
komplexen Ableitungsoperator $d/dz$ übereinstimmt.

Die Definition
\eqref{buch:holomorph:nichtholomorph:eqn:partialddz}
suggeriert, dass man auch den ``konjugiert konplexen'' Operator
\begin{equation}
\frac{\partial}{\partial\bar{z}}
=
\frac12\biggl(
\frac{\partial}{\partial x}
+
i
\frac{\partial}{\partial y}
\biggr)
\end{equation}
definieren kann.
Angewendet auf eine komplexe Funktion $f=u+iv$ lautet er
\[
\frac{\partial f}{\partial \bar{z}}
=
\frac12
\biggl(
\frac{\partial u}{\partial x} + i\frac{\partial v}{\partial x}
+
i\frac{\partial u}{\partial y} - \frac{\partial v}{\partial y}
\biggr)
=
\frac12\biggl(
\frac{\partial u}{\partial x}
-
\frac{\partial v}{\partial y}
\biggr)
+
\frac{i}2\biggl(
\frac{\partial v}{\partial x}
+
\frac{\partial u}{\partial y}
\biggr).
\]
Die Cauchy-Riemann-Differentialgleichungen ermöglichen, dem Real-
und Imaginärteil
\begin{align*}
\frac{\partial u}{\partial x}
&=
\frac{\partial v}{\partial y}
&&\Rightarrow&
\frac{\partial u}{\partial x}-\frac{\partial v}{\partial y}
&=
\operatorname{Re}
\frac{\partial f}{\partial\bar{z}}
=
0
\\
\frac{\partial u}{\partial y}
&=
-\frac{\partial v}{\partial x}
&&\Rightarrow&
\frac{\partial v}{\partial x}
+
\frac{\partial u}{\partial y}
&=
\operatorname{Im}\frac{\partial f}{\partial\bar{z}}
=
0
\end{align*}
einen Sinn zu geben.
Wenn sowohl Realteil wie auch Imaginärteil von $\partial f/\partial\bar{z}$
verschwinden, ist die Funktion holomorph.
Die Notation als partielle Ableitung suggeriert, dass eine holomorphe
Funktion nicht von $\bar{z}$ abhängt, sondern nur von $z$.

\begin{beispiel}
Die Funktion $z\mapsto z$ hat Realteil $u(x,y)=x$ und $v(x,y)=y$
mit den partiellen Ableitungen
\[
\frac{\partial z}{\partial x} = 1 
\quad\text{und}\quad
\frac{\partial z}{\partial y} = i
\qquad\Rightarrow\qquad
\left\{
\begin{aligned}
\frac{\partial z}{\partial z}
&=
\frac12
\biggl(
\frac{\partial z}{\partial x}
-i
\frac{\partial z}{\partial y}
\biggr)
=
\frac{1}{2}
(1+1)
=
1,
\\
\frac{\partial z}{\partial\bar{z}}
&=
\frac12
\biggl(
\frac{\partial z}{\partial x}
+i
\frac{\partial z}{\partial y}
\biggr)
=
\frac12(1-1)
=
0.
\end{aligned}
\right.
\]
Wie erwartet zeigt die verschwindende partielle Ableitung nach $\bar{z}$,
dass die Funktion holomorph ist.
\end{beispiel}

\begin{beispiel}
Die Funktion $z\mapsto\bar{z}$ hat Realteil $u(x,y)=x$ und
Imaginärteil $v(x,y)=-y$ und damit die partiellen Ableitungen
\[
\frac{\partial \bar{z}}{\partial x} = 1 
\quad\text{und}\quad
\frac{\partial \bar{z}}{\partial y} = -i
\qquad\Rightarrow\qquad
\left\{
\begin{aligned}
\frac{\partial \bar{z}}{\partial z}
&=
\frac12
\biggl(
\frac{\partial z}{\partial x}
-i
\frac{\partial z}{\partial y}
\biggr)
=
\frac12 (1-1)
=
0,
\\
\frac{\partial \bar{z}}{\partial \bar{z}}
&=
\frac12
\biggl(
\frac{\partial\bar{z}}{\partial x}
+i
\frac{\partial\bar{z}}{\partial y}
\biggr)
=
\frac12(1+1)
=
1.
\end{aligned}
\right.
\]
Die partielle Ableitung nach $\bar{z}$ verschwindet nicht, also
sind die Cauchy-Riemann-Diffe\-rentialgleichung nicht erfüllt, sie
kann also nicht holomorph sein.
\end{beispiel}

Wir fassen die Erkenntnisse dieses Abschnitts wie folgt zusammen.

\begin{definition}
Die partiellen Ableitungen einer komplexen Funktion $f\colon U\to\mathbb{C}$
mit $U\subset\mathbb{C}$ sind definiert durch die partiellen Ableitungen
nach Realteil $x$ und Imaginärteil $y$ der komplexen Variable $z=x+iy$
durch
\begin{align*}
\frac{\partial}{\partial z}
&=
\frac12\biggl(
\frac{\partial}{\partial x}
-i
\frac{\partial}{\partial y}
\biggr)
&&\text{und}&
\frac{\partial}{\partial\bar{z}}
&=
\frac12\biggl(
\frac{\partial}{\partial x}
+i
\frac{\partial}{\partial y}
\biggr).
\end{align*}
Umgekehrt ist
\[
\frac{\partial}{\partial x}
=
\frac{\partial}{\partial z}
+
\frac{\partial}{\partial\bar{z}}
\qquad\text{und}\qquad
\frac{\partial}{\partial y}
=
i
\biggl(
\frac{\partial}{\partial z}
-
\frac{\partial}{\partial\bar{z}}
\biggr)
\]
Eine Funktion heisst \emph{holomorph}, wenn sie als Funktion zweier
reeller Variablen differenzierbar ist und ausserdem die partielle
Ableitung nach $\bar{z}$
\[
\frac{\partial f}{\partial \bar{z}}
=
0
\]
verschwindet.
\end{definition}

Die konjugiert komplexe Funktion $\bar{f}$ hat die komplexen
Ableitungen
\begin{equation}
\begin{aligned}
\frac{\partial \bar{f}}{\partial z}
&=
\frac12
\biggl(
\frac{\partial u}{\partial x}
-
i
\frac{\partial v}{\partial x}
-
i
\biggl(
\frac{\partial u}{\partial y}
-i
\frac{\partial v}{\partial y}
\biggr)
\biggr)
=
\frac12\biggl(
\frac{\partial u}{\partial x}-\frac{\partial v}{\partial y}
\biggr)
-
\frac{i}2
\biggl(
\frac{\partial u}{\partial y}
+
\frac{\partial v}{\partial x}
\biggr)
=
\overline{\biggl(
\frac{\partial f}{\partial \bar{z}}
\biggr)}
\\
\frac{\partial \bar{f}}{\partial \bar{z}}
&=
\frac12
\biggl(
\frac{\partial u}{\partial x} - i \frac{\partial v}{\partial x}
+
i
\biggl(
\frac{\partial u}{\partial y} - i \frac{\partial v}{\partial y}
\biggr)
\biggr)
=
\frac12
\biggl(
\frac{\partial u}{\partial x}
+
\frac{\partial v}{\partial y}
\biggr)
+
\frac{i}{2}
\biggl(
\frac{\partial u}{\partial y}
-
\frac{\partial v}{\partial x}
\biggr)
=
\overline{\biggl(
\frac{\partial f}{\partial z}
\biggr)}.
\end{aligned}
\label{buch:holomorph:nichtholomorph:eqn:dkonj}
\end{equation}
Formal vertauscht der Konjugationsoperator also mit den Ableitungen,
wenn man die Variable, nach der abgeleitet wird, ebenfalls konjugiert.

%
% Jacobi-Matrix einer komplexen Funktion
%
\subsection{Die komplexe Jacobi-Matrix einer komplexen Funktion}
Die ursprünglichen Ableitungen nach $x$ und $y$ können durch Linearkombination
der beiden partiellen Ableitungsoperatoren $\partial/\partial z$ und
$\partial/\partial\bar{z}$ wiedergewonnen werden:
\begin{align*}
\frac{\partial f}{\partial x}
=
\frac{\partial u}{\partial x}
+
i\frac{\partial v}{\partial x}
&=
\frac{\partial f}{\partial z}
+
\frac{\partial f}{\partial \bar{z}}
&&\text{und}
&
-i
\frac{\partial f}{\partial y}
=
-i\frac{\partial u}{\partial y}
+
\frac{\partial v}{\partial y}
&=
\frac{\partial f}{\partial z}
-
\frac{\partial f}{\partial \bar{z}}
\end{align*}
Die partiellen komplexen Ableitungsoperatoren enthalten also die
gesamte Information, die die partiellen Ableitungen nach den reellen
Variablen $x$ und $y$ wiedergeben.

Auch der Real- und Imaginärteil einer komplexen Zahl
$\Delta z=\Delta x + i\Delta y$
lässt sich gemäss \eqref{buch:holomorph:nichtholomorph:eqn:xyzzbar}
aus $\Delta z$ und $\overline{\Delta z}=\Delta x -i\Delta y$
rekonstruieren lassen.
Die Jacobi-Matrix der Funktion $f$ betrachtet als reelle Funktion
$\mathbb{R}^2\to\mathbb{R}^2$ sollte sich also auch als eine Matrix
Matrix ausdrücken lassen, die auf den 2-dimensionalen komplexen
Vektor mit Komponenten $\Delta z$ und $\overline{\Delta z}$ wirken.

Da wir nicht a priori davon ausgehen möchten, dass die Funktion $f$
holomorph ist, schreiben wir sie als $f(z,\bar{z})$ und untersuchen
später, was sich ergibt, wenn $f$ holomorph ist, also nicht von $\bar{z}$
abhängt.

Die reelle Jacobi-Matrix der Funktion $f(z) = u(x,y) + iv(x,y)$
ermöglicht die Konstruktion der linearen Ersatzfunktion von $f$
betrachtet als Funktion zweier reeller Variablen.
Wir verwenden Vektornotation, um diese wie folgt auszudrücken:
\begin{align*}
\begin{pmatrix}
u(x+\Delta x,y+\Delta y)\\
v(x+\Delta x,y+\Delta y)
\end{pmatrix}
&=
\begin{pmatrix}
u(x,y)\\
v(x,y)
\end{pmatrix}
+
\begin{pmatrix}
\frac{\partial u}{\partial x}(x,y) &
\frac{\partial u}{\partial y}(x,y) \\
\frac{\partial v}{\partial x}(x,y) &
\frac{\partial v}{\partial y}(x,y)
\end{pmatrix}
\begin{pmatrix}
\Delta x \\
\Delta y
\end{pmatrix}
+
o(\Delta x,\Delta y)
\end{align*}
In diesem Ausdruck möchten wir alles durch die Funktionen $f(z,\bar{z})$
und ihre konjugiert komplexe Funktion $\bar{f}(z,\bar{z})$ sowie die Variablen
$z$ und $\bar{z}$ und die partiellen Ableitungen nach $z$ und $\bar{z}$
ausdrücken.
\[
\begin{pmatrix}
\Delta x\\
\Delta y
\end{pmatrix}
=
\begin{pmatrix}
\frac12   (\Delta z+\overline{\Delta z}) \\
\frac1{2i}(\Delta z-\overline{\Delta z})
\end{pmatrix}
\]
\begin{align*}
\frac{\partial u}{\partial x}
&=
\frac12
\biggl(
\frac{\partial f}{\partial x}
+
\frac{\partial \bar{f}}{\partial x}
\biggr)
=
\frac12
\biggl(
\frac{\partial f}{\partial z}
+
\frac{\partial f}{\partial \bar{z}}
+
\frac{\partial \bar{f}}{\partial z}
+
\frac{\partial \bar{f}}{\partial \bar{z}}
\biggr)
\\
\frac{\partial u}{\partial y}
&=
\frac12
\biggl(
\frac{\partial f}{\partial y}
+
\frac{\partial \bar{f}}{\partial y}
\biggr)
=
\frac12
\biggl(
i
\biggl(
\frac{\partial f}{\partial z}
-
\frac{\partial f}{\partial\bar{z}}
\biggr)
+
i
\biggl(
\frac{\partial\bar{f}}{\partial z}
-
\frac{\partial\bar{f}}{\partial\bar{z}}
\biggr)
\biggr)
\\
\frac{\partial v}{\partial x}
&=
\frac1{2i}
\biggl(
\frac{\partial f}{\partial x}
-
\frac{\partial\bar{f}}{\partial x}
\biggr)
=
\frac1{2i}
\biggl(
\frac{\partial f}{\partial z}
+
\frac{\partial f}{\partial \bar{z}}
-
\frac{\partial \bar{f}}{\partial z}
-
\frac{\partial \bar{f}}{\partial \bar{z}}
\biggr)
\\
\frac{\partial v}{\partial y}
&=
\frac1{2i}
\biggl(
\frac{\partial f}{\partial y}
-
\frac{\partial\bar{f}}{\partial y}
\biggr)
=
\frac1{2i}
\biggl(
i\biggl(
\frac{\partial f}{\partial z}
-
\frac{\partial f}{\partial\bar{z}}
\biggr)
-i
\biggl(
\frac{\partial\bar{f}}{\partial z}
-
\frac{\partial\bar{f}}{\partial\bar{z}}
\biggr)
\biggr)
\end{align*}
Im Interesse etwas kompakterer Formeln schreiben wir die partiellen
Ableitungen nach $z$ und $\bar{z}$ in der in der Theorie der partiellen
Differentialgleichungen üblichen Form durch einen Index in der Form
\begin{align*}
\frac{\partial f}{\partial z}
&=
f_z
&\text{und}&
&
\frac{\partial f}{\partial \bar{z}}
&=
f_{\bar{z}}.
\end{align*}
Damit können wir jetzt die lineare Ersatzfunktion als
\begin{align*}
u(x+\Delta x,y+\Delta y)
&\approx
\frac{\partial u}{\partial x}\Delta x
+
\frac{\partial u}{\partial y}\Delta y
\\
&=
\frac12
(
f_z
+
f_{\bar{z}}
+
\bar{f}_z
+
\bar{f}_{\bar{z}}
)
\frac12
(\Delta z + \overline{\Delta z})
+
\frac{i}2
(
f_z
-
f_{\bar{z}}
+
\bar{f}_z
-
\bar{f}_{\bar{z}}
)
\frac1{2i}
(\Delta z - \overline{\Delta z})
\\
&=
\frac12
(
f_z + \bar{f}_z
)
\Delta z
+
\frac12
(
f_{\bar{z}} + \bar{f}_{\bar{z}}
)
\overline{\Delta z}
\\
v(x+\Delta x,y+\Delta y)
&\approx
\frac{\partial v}{\partial x}\Delta x
+
\frac{\partial v}{\partial y}\Delta y
\\
&=
\frac1{2i}
(
f_z + f_{\bar{z}} - \bar{f}_z - \bar{f}_{\bar{z}}
)
\frac12
(\Delta z + \overline{\Delta z})
+
\frac12
(
f_z - f_{\bar{z}} - \bar{f}_z + \bar{f}_{\bar{z}}
)
\frac1{2i}
(\Delta z - \overline{\Delta z})
\\
&=
\frac1{2i}
(
f_z - \bar{f}_z
)
\Delta z
+
\frac1{2i}
(
f_{\bar{z}}
-
\bar{f}_{\bar{z}}
)
\overline{\Delta z}.
\intertext{schreiben, wobei dass Approximationszeichen $\approx$
immer eine Approximation bis auf einen $o$-Term bedeutet.
Die Funktionen $u$ und $v$ sind Real- und Imaginärteil der
Funktion $f$, daher kann man auch}
f(z+\Delta z, \bar{z}+\bar{\Delta z})
&=
u(x+\Delta x, y + \Delta y)
+
iv(x+\Delta x, y + \Delta y)
\\
&=
f_z\Delta z + f_{\bar{z}}\overline{\Delta z}
+
o(\Delta z, \overline{\Delta z})
\\
\bar{f}(z+\Delta z, \bar{z}+\overline{\Delta z})
&=
u(x+\Delta x, y + \Delta y)
-
iv(x+\Delta x, y + \Delta y)
\\
&=
\bar{f}_z \Delta z
+
\bar{f}_{\bar{z}}\Delta z
+
o(\Delta z, \overline{\Delta z})
\end{align*}
schreiben.
In Matrixform ist dies
\begin{align*}
\begin{pmatrix}
     f (z + \Delta z, \bar{z}+\overline{\Delta z})\\
\bar{f}(z + \Delta z, \bar{z}+\overline{\Delta z})
\end{pmatrix}
&=
\bgroup
\renewcommand{\arraystretch}{2.2}
\begin{pmatrix}
\displaystyle\frac{\partial f}{\partial z}(z,\bar{z}) &
\displaystyle\frac{\partial f}{\partial \bar{z}}(z,\bar{z}) \\
\displaystyle\frac{\partial \bar{f}}{\partial z}(z,\bar{z}) &
\displaystyle\frac{\partial \bar{f}}{\partial \bar{z}}(z,\bar{z})
\end{pmatrix}
\begin{pmatrix}
\Delta z\\
\overline{\Delta z}
\end{pmatrix}.
\egroup
\intertext{Die Matrix auf der rechten Seite kann man als eine
komplexe Form der Jacobimatrix sehen.
Wir können die rechte Seite auch als}
\frac{\partial(f,\bar{f})}{\partial(z,\bar{z})}
\begin{pmatrix}
\Delta z \\
\overline{\Delta z}
\end{pmatrix}
\end{align*}
schreiben.

Für eine holomorphe Funktion wird die komplexe Jacobi-Matrix zu
einer Diagonalmatrix.
Man beachte auch, dass wegen die Elemente der unteren Zeile
\eqref{buch:holomorph:nichtholomorph:eqn:dkonj}
durch Konjugation und Vertauschen aus den Elementen der oberen
Zeile hervorgehen.



