%
% imaginaerteil.tex -- template for standalon tikz images
%
% (c) 2021 Prof Dr Andreas Müller, OST Ostschweizer Fachhochschule
%
\documentclass[tikz]{standalone}
\usepackage{amsmath}
\usepackage{times}
\usepackage{txfonts}
\usepackage{pgfplots}
\usepackage{csvsimple}
\usetikzlibrary{arrows,intersections,math,calc}
\begin{document}
\definecolor{darkred}{rgb}{0.8,0,0}
\def\skala{1}
\def\punkt#1#2{
	\fill[color=white] #1 circle[radius=0.05];
	\draw[color=#2] #1 circle[radius=0.05];
}
\begin{tikzpicture}[>=latex,thick,scale=\skala]

\clip (-0.45,-0.45) rectangle ++(12.6,6.6);

\draw[->] (-0.1,0) -- (11.9,0) coordinate[label={$\operatorname{Re}$}];
\draw[->] (0,-0.1) -- (0,6) coordinate[label={right:$\operatorname{Im}$}];

\coordinate (A) at (1.5,1.5);
\coordinate (B) at (7.0,1.5);
\coordinate (C) at (7.0,4.5);

\draw[line width=0.3pt] (A) -- (1.5,0);
\draw (1.5,-0.05) -- ++(0,0.1);
\node at (1.5,0) [below] {$x_0\mathstrut$};

\draw[line width=0.3pt] (A) -- (0,1.5);
\draw (-0.05,1.5) -- ++(0.1,0);
\node at (0,1.5) [left] {$y_0\mathstrut$};

\draw[line width=0.3pt] (B) -- (7,0);
\draw (7,-0.05) -- ++(0,0.1);
\node at (7,0) [below] {$x\mathstrut$};

\draw[line width=0.3pt] (C) -- (0,4.5);
\draw (-0.05,4.5) -- ++(0.1,0);
\node at (0,4.5) [left] {$y\mathstrut$};

\draw[color=darkred,line width=1.4pt] (A) -- (B);
\draw[color=blue,line width=1.4pt] (B) -- (C);
\punkt{(A)}{darkred}
\punkt{(B)}{violet}
\punkt{(C)}{blue}

\node at (A) [above] {$v(x_0,y_0)=v_0$};
\node at (B) [right] {$v(x,y_0)$};
\node at (C) [above ] {$v(x,y)$};

\node[color=darkred] at ($0.5*(A)+0.5*(B)$) [below] {$\displaystyle
	\int_{x_0}^x \frac{\partial v}{\partial x}(\xi,y_0)\,d\xi
	=
	-\int_{x_0}^x \frac{\partial u}{\partial y}(\xi,y_0)\,d\xi
$};
\node[color=blue] at ($0.5*(B)+0.5*(C)$) [right] {$\displaystyle
	\int_{y_0}^y \frac{\partial v}{\partial y}(x,\eta)\,d\eta
	=
	\int_{y_0}^y \frac{\partial u}{\partial x}(x,\eta)\,d\eta
$};

\end{tikzpicture}
\end{document}

