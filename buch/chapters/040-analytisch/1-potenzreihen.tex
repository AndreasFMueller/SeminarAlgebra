%
% 1-potenzreihen.tex -- Potenzreihenentwicklung holomorpher Funktionen
%
% (c) 2025 Prof Dr Andreas Müller
%
\section{Potenzreihen holomorpher Funktionen
\label{buch:analytisch:section:potenzreihen}}
\kopfrechts{Potenzreihen holomorpher Funktionen}
Der Residuensatz stellt einen Zusammenhang zwischen $f(z)$ und den
Werten von $f(x)/(x-a)$ auf einem geschlossenen Weg her.
Der Nenner kan mit Hilfe der geometrischen Reihe in eine Potenzreihe
verwandelt werden, die sich zu einer Potenzreihe von $f(z)$ umformen
lässt.

%
% Die geometrische Reihe
%
\subsection{Die geometrische Reihe}
Die geometrische Reihe ist die Reihe
\[
a+aq+aq^2 + \dots = \sum_{k=0}^\infty aq^k.
\]
Es ist wohlbekannt, dass diese Reihe für $q<1$ konvergent ist und
die Summe
\[
\sum_{k=0}^n aq^k
=
a\frac{q^{n+1}-1}{q-1}
\qquad\Rightarrow\qquad
\sum_{k=0}^n aq^k
=
\lim_{n\to\infty}
a\frac{q^{n+1}-1}{q-1}
=
a\frac{1}{1-q}.
\]

Für komplexe Zahlen $z\in\mathbb{C}$ mit $|z|<1$ definiert die
geometrische Reihe die holomorphe Funktion
\[
\sum_{k=1} z^k
=
\frac{1}{1-z}
=
f(z).
\]
Der Konvergenzradius der Reihe ist $\varrho=1$.
Die Polstelle bei $z=1$ führt zu einer divergenten Potenzreihe.

Die Funktion 
\[
z
\mapsto
\frac{1}{x-z\mathstrut}
\]
ist ebenfalls holomorph, wir möchten sie jetzt in eine im Punkt $a$
zentrierte Potenzreihe entwickeln.
Dazu verwenden wir die Identität
\begin{equation}
\frac{1}{x-z\mathstrut}
=
\frac{1}{(x-a)-(z-a)\mathstrut}
=
\frac{1}{\displaystyle(x-a)\biggl(1-\frac{z-a\mathstrut}{x-a\mathstrut}\biggr)}
=
\frac{1}{x-a}
\cdot
\frac{1}{\displaystyle1-\frac{z-a\mathstrut}{x-a\mathstrut}}.
\label{buch:analytisch:eqn:geometrisch1}
\end{equation}
Der letzte Faktor auf der rechten Seite lässt sich mit der geometrischen
Reihe als Potenzreihe der Form
\begin{equation}
\frac{1}{\displaystyle1-\frac{z-a\mathstrut}{x-a\mathstrut}}.
=
\sum_{k=0}^\infty \biggl(\frac{z-a\mathstrut}{x-a\mathstrut}\biggr)^k
=
\sum_{k=0}^\infty
c_k
(z-a)^k
\qquad\text{mit $c_k=1/(x-a)^k$.}
\label{buch:analytisch:eqn:geometrischintegriert}
\end{equation}
entwickeln.

%
% Integration der geometrischen Reihe
%
\subsection{Integration der geometrischen Reihe}
Der Residuensatz ermöglich, mit Hilfe der geometrischen Reihe eine
Potenzreihenentwicklung für die die Funktion $f(z)$ zu finden.
Dazu erinnern wir uns daran, dass 
\[
f(z)
=
\oint_{\gamma}
\frac{f(x)}{x-z\mathstrut}
\,dx,
\]
wenn $\gamma$ ein Weg ist, der den Punkt $z$ im Gegenuhrzeigersinn
einmal umwindet.
Jetzt verwenden wir
\eqref{buch:analytisch:eqn:geometrisch1}
und
\eqref{buch:analytisch:eqn:geometrischintegriert}
und erhalten
\begin{align}
f(z)
&=
\frac{1}{2\pi i}
\oint_\gamma
\frac{1}{x-a}
\cdot
\frac{1}{\displaystyle1-\frac{z-a\mathstrut}{x-a\mathstrut}}
\,dx
\notag
\\
&=
\frac{1}{2\pi i}
\oint_\gamma
\frac{1}{x-a}
\sum_{k=0}^\infty
f(x)
\biggl(
\frac{z-a\mathstrut}{x-a\mathstrut}
\biggr)^k
\,dx
\notag
\\
&=
\sum_{k=0}^\infty
\biggl(
\frac{1}{2\pi i}
\oint_\gamma
\frac{f(x)}{(x-a)^{k+1}}
\,dx
\biggr)
(z-a)^k.
\label{buch:analytisch:eqn:reihe}
\end{align}
Damit ist eine Potenzreihe im Punkt $a$ für die Funktion
$f(z)$ gefunden.
Sie ist konvergent, solange 
\[
\biggl|
\frac{z-a}{x-a}
\biggr|
=
\frac{|z-a|}{|x-a|}
<
1
\qquad\Rightarrow\qquad
|z-a| < |x-a|
\]
für alle $x$ entlang des Weges $\gamma$ gilt.
Dies wird erreicht, wenn 
\[
|z-a| < \inf_{\gamma} |x-a|,
\]
auf der rechten Seite steht der kleinste Abstand des Punktes $a$ vom
Weg $\gamma$.

%
% Potenzreihenentwicklung einer holomorphen Funktion
%
\subsection{Potenzreihenentwicklung einer holomorphen Funktion}
Falls $\gamma$ ein geschlossener Weg ist, der sich genau einmal
im Gegenuhrzeigersinn um den Punkt $a$ windet, lassen sich die
Integrale auf der rechten Seite von \eqref{buch:analytisch:eqn:reihe}
können mit Satz~\ref{buch:integration:satz:ableitungn}
berechnet.
Sie ergeben
\begin{align*}
f(z)
&=
\sum_{k=0}^\infty
\frac{f^{(k)}(a)}{k!}
(z-a)^k.
\end{align*}
Damit ist die gefundene Potenzreihe der Funktion $f$ die
Taylor-Reihe der Funktion $f$ im Punkt $a$.

\begin{satz}
\label{buch:analytisch:satz:holomorph-analytisch}
Eine holomorphe Funktion $f(z)$ ist analytisch, die Taylor-Reihe
von $f$ im Punkt $a$ konvergiert.
Der Konvergenzradius ist das Supremum der Radien von Kreisen um $a$,
die vollständig im Definitionsbereich von $f$ enthalten sind.
\end{satz}

%
% Reelle differenzierbare Funktionen sind nicht alle analytisch
%
\subsection{Reelle differenzerierbare Funktionen sind nicht alle analytisch}
Die Aussage des Satzes~\ref{buch:analytisch:satz:holomorph-analytisch}
hat keine Entsprechung für reelle beliebig oft differenzierbare
Funktionen.
Vielmehr gibt es beliebig oft differenzierbare Funktion reellwertige
Funktionen auf $\mathbb{R}$, die keine konvergente Potenzreihe haben.

%
% Eine glatte Funktion mit verschwindender Potenzreihe
%
\subsubsection{Eine glatte Funktion mit verschwindender Potenzreihe}
Wir betrachten die Funktion
\[
f(x)
=
\begin{cases}
e^{-1/x}&\qquad \text{für $x>$}\\
0&\qquad\text{sonst.}
\end{cases}
\]
Sie ist beliebig oft differenzierbar.
Die Ableitung ist ausserhalb des Punktes $x=0$ stetig, es muss also
nur noch untersucht werden, dass die Ableitung auch im Punkt $x=0$
stetig ist.

Die Ableitung von $f(x)$ ist
\begin{equation}
f'(x)
=
\frac{1}{x^2}
e^{-1/x}
\label{buch:analytisch:eqn:f'}
\end{equation}
für $x>0$.
Für höhere Ableitungen wird der Ausdruck komplizierter, wir wollen
mit vollständiger Induktion zeigen, dass alle Ableitungen von der Form
\begin{equation}
f^{(n)}(x)
=
P_n\biggl(\frac1x\biggr) e^{-1/x}
\label{buch:analytisch:eqn:fn=Pne}
\end{equation}
ist, wobei $P_n$ ein Polynom ist.
Die Gleichung \eqref{buch:analytisch:eqn:f'} dient dazu als
Induktionsverankerung.

Für den Induktionsschritt nehmen wir jetzt an, dass 
\eqref{buch:analytisch:eqn:fn=Pne}
für die $n$-te Ableitung gilt und zeigen, dass auch die $n+1$-te
Ableitung diese Form hat.
Dazu berechnen wir
\begin{align*}
f^{(n+1)}(x)
&=
\frac{d}{dx}
f^{(n)}(x)
=
\frac{d}{dx} P\biggl(\frac1x\biggr) e^{-1/x}
\\
&=
-P'_n\biggl(\frac1x\biggr) \frac{1}{x^2} e^{-1/x}
+
P_n\biggl(\frac1x\biggr) \frac{1}{x^2} e^{-1/x}
\\
&=
\frac{1}{x^2}
\biggl(
P_n\biggl(\frac1x\biggr)
-
P_n'\biggl(\frac1x\biggr)
\biggr)e^{-1/x}
\end{align*}
Setzt man 
\begin{equation}
P_{n+1}(X)
=
X^2(P_n(X) - P'_n(X)),
\label{buch:analytisch:eqn:e1rekursion}
\end{equation}
kann man tatsächlich
\[
f^{(n+1)}(x)
=
P_{n+1}\biggl(\frac1x\biggr)e^{-1/x}
\]
schreiben.
Die Rekursionsformel~\eqref{buch:analytisch:eqn:e1rekursion}
liefert ein Polynom $P_{n+1}$, womit der Induktionsschritt vollzogen
ist.

Die Polynome $P_n$ aus dem vorangegangenen Beweis lassen sich
mit der Rekursionsformel~\eqref{buch:analytisch:eqn:e1rekursion}
sehr leicht berechnen, die ersten paar Polynome sind
\begin{align*}
P_0(X) &= 1 \\
P_1(X) &= X^2 \\
P_2(X) &= X^4 - 2X^3 \\
P_3(X) &= X^2 (X^4 - 2X^3 - 4X^3 + 6X^2) = X^6 -6X^5+6X^4 \\
P_4(X) &= X^8 -12X^7+36X^6-24X^5 \\
P_5(X) &= X^{10} -20X^9 +120 X^8 - 240X^7 + 120 X^6.
\end{align*}
Da für $x\to 0$ der Exponentialfaktor $e^{-1/x}$ immer schneller
gegen $0$ geht, als ein polynomialer Faktor $P_n(1/x)$ gegen $\infty$
gehen kann, ist der Grenzwert 
\[
\lim_{x\to 0+}
f^{(n)}(x)
=
\lim_{\xi\to \infty}
P_n(\xi) e^{-\xi}
=
0.
\]
Damit ist gezeigt, dass an der Stelle $x=0$ alle Ableitungen von $f$
stetig sind, $f$ ist eine glatte Funktion.

Da die Ableitungen an der Stelle $x=0$ verschwinden, ist die
Taylor-Reihe von $f$ die Nullfunktion.
Da $f$ selbst nicht die Nullfunktion ist, kann die Taylor-Reihe
nicht gegen $f$ konvergieren.

Die Funktion $f(x)$ lässt sich nicht zu einer holomorphen Funktion
ausdehnen.
Der einzige dafür in Frage kommende Kandidat ist die Funktion
$z\mapsto e^{-1/z}$.
Nähert sich $z$ auf der imaginären Achse dem Punkt $0$, zum Beispiel
mit $z=i/t$ für $t\to \infty$, dann ist
\[
\lim_{\varepsilon\to 0} e^{-1/i\varepsilon}
=
\lim_{t\to\infty} e^{it}
=
\lim_{t\to\infty}(\cos t + i\sin t).
\]
Der Grenzwert auf der rechten Seite konvergiert allerdings nicht, sodass
die Funktion $z\mapsto e^{-1/z}$ im Punkt $z=0$ nicht einmal stetig
sein kann.




