%
% nichtanalytisch.tex -- Plot einer nicht analytischen Funktion
%
% (c) 2021 Prof Dr Andreas Müller, OST Ostschweizer Fachhochschule
%
\documentclass[tikz]{standalone}
\usepackage{amsmath}
\usepackage{times}
\usepackage{txfonts}
\usepackage{pgfplots}
\usepackage{csvsimple}
\usetikzlibrary{arrows,intersections,math}
\definecolor{darkred}{rgb}{0.8,0,0}
\begin{document}
\def\skala{1}
\begin{tikzpicture}[>=latex,thick,scale=\skala]

\draw[->] (-3.05,0) -- (8.5,0) coordinate[label={$x$}];
\draw[->] (0,-0.1) -- (0,4.4) coordinate[label={right:$f(x)$}];
\draw[line width=0.2pt] (-3,4) -- (8,4);

\begin{scope}
\clip (-3,-0.1) rectangle (8,4);
\draw[color=darkred,line width=1.4pt]
	(-3,0) -- (0,0) --
	plot[domain=0.001:0.67,samples=100] ({-4/ln(\x)},{4*\x});
\end{scope}

\node[color=darkred] at (4,2) {$f(x)=e^{-1/x}$};
\node[color=darkred] at (-1.5,0) [above] {$f(x)=0$};

\node at (0,-0.05) [below] {$0$};
\node at (4,-0.05) [below] {$1$};
\node at (8,-0.05) [below] {$2$};
\draw (4,-0.05) -- ++(0,0.1);
\draw (8,-0.05) -- ++(0,0.1);
\draw (-0.05,4) -- ++(0.1,0);
\node at (-0.05,4) [below left] {$1$};

\end{tikzpicture}
\end{document}

