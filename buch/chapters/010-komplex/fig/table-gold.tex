%
% table-gold.tex
%
% (c) 2026 Prof Dr Andreas Müller
%
\begin{table}
\centering
\begin{tabular}{|>{$}r<{$}>{$}l<{$}|>{$}r<{$}|}
\hline
n & y_n & x_n \\
\hline
 1 & 2                 & 0.508034707497476
\\
 2 & 1.500000000000000             & -0.730167373438399 \\
 3 & 1.\underline{6}66666666666667 &  0.319690815934174 \\
 4 & 1.\underline{6}00000000000000 & -1.404165739925735 \\
 5 & 1.\underline{6}25000000000000 & -0.345999548896763 \\
 6 & 1.\underline{61}5384615384615 &  1.272088814812140 \\
 7 & 1.\underline{61}9047619047619 &  0.242990090617789 \\
 8 & 1.\underline{61}7647058823529 & -1.936202034966178 \\
 9 & 1.\underline{618}181818181818 & -0.709863503540628 \\
10 & 1.\underline{61}7977528089888 &  0.349429012666962 \\
11 & 1.\underline{6180}55555555556 & -1.256191291052456 \\
12 & 1.\underline{6180}25751072961 & -0.230067094013908 \\
13 & 1.\underline{61803}7135278515 &  2.058245522488201 \\
14 & 1.\underline{61803}2786885246 &  0.786197418015100 \\
15 & 1.\underline{61803}4447821682 & -0.242873870579829 \\
16 & 1.\underline{618033}813400125 &  1.937244794475075 \\
17 & 1.\underline{61803}4055727554 &  0.710523884635529 \\
18 & 1.\underline{6180339}63166706 & -0.348444169203709 \\
19 & 1.\underline{6180339}98521804 &  1.260728028475482 \\
20 & 1.\underline{61803398}5017358 &  0.233767770871423 \\
21 & 1.\underline{6180339}90175597 & -2.021990939507587 \\
22 & 1.\underline{618033988}205325 & -0.763714440828034 \\
23 & 1.\underline{618033988}957902 &  0.272837745754085 \\
24 & 1.\underline{618033988}670443 & -1.696172136911838 \\
25 & 1.\underline{6180339887}80243 & -0.553304666781480 \\
26 & 1.\underline{6180339887}38303 &  0.627008940439556 \\
27 & 1.\underline{6180339887}54322 & -0.483932324938967 \\
28 & 1.\underline{61803398874}8204 &  0.791236155774208 \\
29 & 1.\underline{6180339887}50541 & -0.236304511027909 \\
30 & 1.\underline{618033988749}648 &  1.997761646531475 \\
\hline
\infty & 1.618033988749895 & \\
\hline
\end{tabular}
\caption{Approximation der Zahl $\varphi=(1+\!\sqrt{5})/2$
durch die Folge $(y_n)_{n\in\mathbb{N}}$, die durch $y_0=1$ und
$y_{n+1}=1+1/y_n$ rekursiv definiert ist (Spalte $y_n$).
In der Spalte $x_n$ sind die Newton-Approximationen für die
Funktion $f(x) = x^2 + 1$, rekursiv definiert durch
$x_{n+1} = x_n -  f(x_n)/f'(x_n)$, dargestellt, die keinerlei
Anzeichen von Konvergenz zeigen.
\label{buch:komplex:zahlen:table:gold}}
\end{table}
