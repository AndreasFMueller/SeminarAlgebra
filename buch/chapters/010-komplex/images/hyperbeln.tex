%
% hyperbeln.tex -- template for standalon tikz images
%
% (c) 2021 Prof Dr Andreas Müller, OST Ostschweizer Fachhochschule
%
\documentclass[tikz]{standalone}
\usepackage{amsmath}
\usepackage{times}
\usepackage{txfonts}
\usepackage{pgfplots}
\usepackage{csvsimple}
\usetikzlibrary{arrows,intersections,math}
\begin{document}
\def\skala{0.82}
\definecolor{darkred}{rgb}{0.8,0,0}
\begin{tikzpicture}[>=latex,thick,scale=\skala,
declare function = {
	X(\u,\t) = \u*cosh(\t);
	Y(\u,\t) = \u*sinh(\t);
	XX(\u,\t) = \u*sinh(\t);
	YY(\u,\t) = \u*cosh(\t);
	RX(\u,\t) = (sqrt(2)/2)*(X(\u,\t)-Y(\u,\t));
	RY(\u,\t) = (sqrt(2)/2)*(X(\u,\t)+Y(\u,\t));
}]

\begin{scope}[xshift=-4.5cm]
	\foreach \x in {-3,-2.5,...,3}{
		\draw[color=blue!10,line width=0.5pt] (\x,-3) -- (\x,3);
	}
	\foreach \x in {-3,-2,...,3}{
		\draw[color=blue!10] (\x,-3) -- (\x,3);
	}
	\foreach \y in {-3,-2.5,...,3}{
		\draw[color=darkred!10,line width=0.5pt] (-3,\y) -- (3,\y);
	}
	\foreach \y in {-3,-2,...,3}{
		\draw[color=darkred!10] (-3,\y) -- (3,\y);
	}
	\begin{scope}
		\clip (-3,-3) rectangle (3,3);
		\foreach \u in {0.5,1.5,...,9}{
			\pgfmathparse{sqrt(\u)}
			\xdef\U{\pgfmathresult}
			\draw[color=blue,line width=0.4pt]
				plot[domain=-2:2,samples=50]
					({X(\U,\x)},{Y(\U,\x)});
			\draw[color=blue,line width=0.4pt]
				plot[domain=-2:2,samples=50]
					({-X(\U,\x)},{Y(\U,\x)});
			\draw[color=blue,line width=0.4pt]
				plot[domain=-2:2,samples=50]
					({XX(\U,\x)},{YY(\U,\x)});
			\draw[color=blue,line width=0.4pt]
				plot[domain=-2:2,samples=50]
					({XX(\U,\x)},{-YY(\U,\x)});
		}
		\foreach \u in {0.5,1.5,...,18}{
			\pgfmathparse{sqrt(\u)}
			\xdef\U{\pgfmathresult}
			\draw[color=darkred,line width=0.4pt]
				plot[domain=-2:2,samples=50]
					({RX(\U,\x)},{RY(\U,\x)});
			\draw[color=darkred,line width=0.4pt]
				plot[domain=-2:2,samples=50]
					({-RX(\U,\x)},{RY(\U,\x)});
			\draw[color=darkred,line width=0.4pt]
				plot[domain=-2:2,samples=50]
					({RX(\U,\x)},{-RY(\U,\x)});
			\draw[color=darkred,line width=0.4pt]
				plot[domain=-2:2,samples=50]
					({-RX(\U,\x)},{-RY(\U,\x)});

		}
		\foreach \u in {1,...,9}{
			\pgfmathparse{sqrt(\u)}
			\xdef\U{\pgfmathresult}
			\draw[color=blue]
				plot[domain=-2:2,samples=50]
					({X(\U,\x)},{Y(\U,\x)});
			\draw[color=blue]
				plot[domain=-2:2,samples=50]
					({-X(\U,\x)},{Y(\U,\x)});
			\draw[color=blue]
				plot[domain=-2:2,samples=50]
					({XX(\U,\x)},{YY(\U,\x)});
			\draw[color=blue]
				plot[domain=-2:2,samples=50]
					({XX(\U,\x)},{-YY(\U,\x)});
		}
		\foreach \u in {0,1,...,18}{
			\pgfmathparse{sqrt(\u)}
			\xdef\U{\pgfmathresult}
			\draw[color=darkred]
				plot[domain=-2:2,samples=50]
					({RX(\U,\x)},{RY(\U,\x)});
			\draw[color=darkred]
				plot[domain=-2:2,samples=50]
					({-RX(\U,\x)},{RY(\U,\x)});
			\draw[color=darkred]
				plot[domain=-2:2,samples=50]
					({RX(\U,\x)},{-RY(\U,\x)});
			\draw[color=darkred]
				plot[domain=-2:2,samples=50]
					({-RX(\U,\x)},{-RY(\U,\x)});

		}
	\end{scope}
	\draw[->] (-3,0) -- (3.3,0) coordinate[label={$\operatorname{Re}$}];
	\draw[->] (0,-3) -- (0,3.3)
		coordinate[label={right:$\operatorname{Im}$}];
\end{scope}

\draw[->] (-0.7,0) -- (0.7,0);
\node at (0,0) [above] {$f(z)$};
\node at (0,0) [below] {$z^2$};

\begin{scope}[xshift=4cm]
	\foreach \x in {-3,-2.5,...,3}{
		\draw[color=blue,line width=0.5pt] (\x,-3) -- (\x,3);
	}
	\foreach \x in {-3,-2,...,3}{
		\draw[color=blue] (\x,-3) -- (\x,3);
	}
	\foreach \y in {-3,-2.5,...,3}{
		\draw[color=darkred,line width=0.5pt] (-3,\y) -- (3,\y);
	}
	\foreach \y in {-3,-2,...,3}{
		\draw[color=darkred] (-3,\y) -- (3,\y);
	}

	\draw[->] (-3,0) -- (3.3,0) coordinate[label={$\operatorname{Re}$}];
	\draw[->] (0,-3) -- (0,3.3)
		coordinate[label={right:$\operatorname{Im}$}];
\end{scope}


\end{tikzpicture}
\end{document}

