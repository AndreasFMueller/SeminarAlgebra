%
% gausszahlen.tex -- template for standalon tikz images
%
% (c) 2021 Prof Dr Andreas Müller, OST Ostschweizer Fachhochschule
%
\documentclass[tikz]{standalone}
\usepackage{amsmath}
\usepackage{times}
\usepackage{txfonts}
\usepackage{pgfplots}
\usepackage{csvsimple}
\usetikzlibrary{arrows,intersections,math}
\begin{document}
\def\skala{1}
\definecolor{darkred}{rgb}{0.8,0,0}
\begin{tikzpicture}[>=latex,thick,scale=\skala]

\foreach \x in {-5,-4,...,5}{
	\draw[color=gray!40] (\x,-4.5) -- (\x,4.5);
}
\foreach \y in {-4,-3,...,4}{
	\draw[color=gray!40] (-5.5,\y) -- (5.5,\y);
}

\draw[->] (-5.5,0) -- (5.5,0) coordinate[label={$\operatorname{Re}$}];
\draw[->] (0,-4.5) -- (0,4.5) coordinate[label={right:$\operatorname{Im}$}];

\draw[->,color=darkred,line width=1.4pt] (0,0) -- (4,3);
\node[color=darkred] at (3.2,2.25) [above left] {$a+bi$};

\draw[->,color=blue,line width=1.4pt] (0,0) -- (0,1);
\node[color=blue] at (-0.0,1) [above left] {$i$};

\draw[->,color=blue,line width=1.4pt] (0,0) -- (1,0);
\node[color=blue] at (1,0) [below right] {$1$};

\foreach \x in {-5,-4,...,5}{
	\foreach \y in {-4,-3,...,4}{
		\fill[color=white] (\x,\y) circle[radius=0.05];
		\draw (\x,\y) circle[radius=0.05];
	}
}

\end{tikzpicture}
\end{document}

