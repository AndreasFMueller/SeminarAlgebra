%
% 1-zahlen.tex -- 1. Komplexe Zahlen
%
% (c) 2025 Prof Dr Andreas Müller
%
\section{Komplexe Zahlen
\label{buch:kmoplex:section:zahlen}}
\kopfrechts{Komplexe Zahlen}
Die Lösungsformel für die quadratische Gleichung
\begin{equation}
x^2 + q = px
\qquad\text{oder}\qquad
x^2 - px + q = 0
\label{buch:komplex:zahlen:eqn:qeq}
\end{equation}
war bereits in babylonischer Zeit etwa 2000 BC bekannt.
Auf Keilschrifttafeln aus der dritten Dynastie von Ur findet man den
folgenden Algorithus:
\begin{enumerate}
\item Berechne die Hälfte von $p$: $\displaystyle\frac{p}{2}$.
\item Quadriere das Resultat: $\displaystyle\biggl(\frac{p}{2}\biggr)^2$.
\item Subtrahiere $q$: $\displaystyle\biggl(\frac{p}2\biggr)^2-q$.
\item Finde die (positive) Quadratwurzel mithilfe einer Tabelle von
Quadratzahlen:
\[
\!\sqrt{\rlap{\phantom{\bigg|}}\smash{\biggl(\frac{p}2\biggr)^2-q}}.
\]
\item Addiere die Resultate der Schritte 1 und 4:
$\displaystyle\frac{p}2+\!\sqrt{\rlap{\phantom{\bigg|}}\smash{\biggl(\frac{p}2\biggr)^2-q}}$.
\end{enumerate}
Der Algorithmus trägt sowohl in der
Formulierung~\eqref{buch:komplex:zahlen:eqn:qeq}
wie auch in der Durchführung der Tatsache Rechnung, dass die
Babylonier die negativen Zahlen noch nicht kannten.
Vor allem im Schritt 4 wird deutlich, dass nur positive Quadratwurzeln
in Betracht gezogen wurden.
Dies hat zur Folge, dass der Algorithmus nur eine Lösung finden kann,
selbst wenn wie im Falle der Gleichung
\begin{equation}
x^2 + 3 = 4x
\qquad
\Rightarrow
\qquad
p=4,\;q=3
\label{buch:komplex:zahlen:eqn:qepos}
\end{equation}
zwei positive, ganzzahlige Lösungen existieren.
Im Falle der Gleichung~\eqref{buch:komplex:zahlen:eqn:qepos}
ergibt der Algorithmus
\begin{align*}
x
&=
\frac{p}2
+
\!\sqrt{\rlap{$\phantom{\bigg|}$}\smash{\biggl(\frac{p}{2}\biggr)^2-q}}
=
\frac{4}{2}
+
\!\sqrt{\rlap{$\phantom{\biggl|}$}\smash{\biggl(\frac{4}{2}\biggr)^2-3}}
=
2
+
\!\sqrt{2^2-3}
=
2
+
\!\sqrt{1}
=
3.
\end{align*}
Die moderne Lösungsformel liefert dagegen die zwei Lösungen
\begin{align*}
x_\pm
&=
\frac{p}2
\pm
\!\sqrt{\rlap{$\phantom{\bigg|}$}\smash{\biggl(\frac{p}{2}\biggr)^2-q}}
=
\frac{4}{2}
\pm
\!\sqrt{\rlap{$\phantom{\biggl|}$}\smash{\biggl(\frac{4}{2}\biggr)^2-3}}
=
2\pm1
=
\begin{cases}
3\\
1.
\end{cases}
\end{align*}
Beide Lösung erfüllen ganz offensichtlich die ursprüngliche
Gleichung
\begin{align*}
x_+^2 + 3 &= 4x_+
&&\Leftrightarrow&
3^2 + 3 &= 4\cdot 3
&&\Leftrightarrow&
12&=12,
\\
x_-^2 + 3 &= 4x_-
&&\Leftrightarrow&
1^2 + 3 &= 4\cdot 1
&&\Leftrightarrow&
4&=4,
\end{align*}
auch wenn die zweite Lösung nicht vom Algorithmus gefunden werden kann.

In Schritt~4 wird verlangt, dass die Quadratwurzel mithilfe einer
Tabelle von Quadratzahlen gefunden werden muss.
Es können also nur Lösungen gefunden werden, für die die Quadratwurzel
im Schritt~4 eine in babylonischer Zeit bekannte Zahl ergab, also
im kompliziertesten Fall ein Bruch oder, wie wir heute sagen würden,
eine rationale Zahl.
Die Gleichung
\begin{align}
x^2 + 4x &= 2
&&\Rightarrow&
p&=4,\; q=2
&&\Rightarrow&
x
&=
\frac{p}{2}
+
\!\sqrt{\rlap{$\phantom{\bigg|}$}\smash{\biggl(\frac{p}2\biggr)^2-q}}
\label{buch:komplex:zahlen:eqn:qeunloesbar}
\\
&&&&&&&&
&=
\frac{4}{2}
+
\!\sqrt{\rlap{$\phantom{\bigg|}$}\smash{\biggl(\frac{4}2\biggr)^2-2}}
\notag
\\
&&&&&&&&
&=
2
+
\!\sqrt{2^2-2}
=
2
+
\!\sqrt{2}
\notag
\end{align}
Die Zahl $2$ kommt in keiner Tabelle von Quadratzahlen vor.
Die Gleichung~\eqref{buch:komplex:zahlen:eqn:qeunloesbar} ist daher für die
babylonische Mathematik unlösbar.

Die moderne Mathematik muss also das Problem lösen, den zur Verfügung
stehenden Zahlenbereich so zu erweitern, dass Gleichungen
wie \eqref{buch:komplex:zahlen:eqn:qeunloesbar} lösbar sind.
Die genauere Inspektion der ursprünglichen
Gleichung~\eqref{buch:komplex:zahlen:eqn:qeq} fördert noch
weitere Schwierigkeiten des Lösungsalgorithmus zu Tage, zum Beispiel
den Fall $p=0$, $q=1$, der auf die Lösung $x=\sqrt{-1}$ führt.
in diesem Abschnitt sollen daher Techniken entwickelt werden, wie die
rationalen Zahlen so erweitert werden können, dass beliebige polynomiale
Gleichungen gelöst werden können.

%
% Erweiterungen von Q
%
\subsection{Erweiterungen von $\mathbb{Q}$}
Schon im Altertum war bekannt, dass die Quadratwurzel $\!\sqrt{2}$ von 
$2$ nicht rational ist.
Diese Entdeckung, die normalerweise den Pythatgoreern zugeschrieben wird,
widersprach der verbreiteten Ansicht, dass sich alle Zahlen durch
Brüche ausdrücken lassen.
In diesem Abschnitt werden zwei Konstruktionen betrachtet, mit denen
sich eine Erweiterung von $\mathbb{Q}$ finden lässt, die $\!\sqrt{2}$
enthält.

%
% Die reellen Zahlen
%
\subsubsection{Die reellen Zahlen}
Die Quadratwurzel $\!\sqrt{2}$ ist bei weitem nicht die einzige irrationale
Zahl.
Jede ganze Zahl, deren Primfaktorzerlegung einen Primfaktor in ungerader
Anzahl enthält, hat eine irrationale Quadratwurzel.
Auch für viele kubische Wurzeln, die eulersche Zahl $e$ oder die
Kreiszahl $\pi$ lassen sich gute Approximationen durch Brüche geben.
Leonhard Euler hat beretis 1737 bewiesen, dass $e$ irrational ist.
Für $\pi$ ist dies erst Johann Heinrich Lambert 1761 gelungen.
Diese Beispiele deuten darauf hin, dass es eine grosse Menge von
Zahlen gibt, die beliebig genau durch Brüche approximiert werden
können, aber trotzdem keine Brüche sind.

Das Problem kann gelöst werden, indem man eine Menge $\mathbb{R}$ von
Zahlen definiert, die beliebig genau durch Brüche approximiert werden
können.
Dabei muss aber auch berücksichtigt werden, dass eine solche Zahl
verschiedene Approximationen haben kann.
Zum Beispiel kann $\pi$ sowohl durch Dezimalbrüche
\[
3
, 3.1
, 3.14
, 3.141
, 3.1415
, 3.14159
, 3.141592
, 3.1415926
, 3.14159265
%, 3.141592659
%, 3.1415926592
%, 3.14159265926
\to
\pi
\]
als auch durch die Kettenbruchentwicklung
\[
\frac{3}{1}
\approx
3.{\color{gray}00},
\frac{22}{7} \approx 3.14{\color{gray}28},
\frac{333}{106}\approx 3.1415{\color{gray}09},
\frac{355}{113}\approx 3.141592{\color{gray}92}
\to
\pi
\]
approximiert werden.
Es reicht daher nicht, eine Approximation zu haben.
Es wird ausserdem ein Kriterium benötigt, mit welchem man entscheiden
kann, ob zwei Approximationen die gleiche Zahl approximieren.

\begin{definition}[Cauchy-Folge]
Eine Folge $(a_n)_{n\in\mathbb{N}}$ von Zahlen $a_n\in\mathbb{Q}$ heisst
\emph{Cauchy-Folge}, wenn für jedes $\varepsilon>0$ ein $N\in\mathbb{N}$
\index{Cauchy-Folge}%
gibt derart, dass
\[
|a_n-a_m| < \varepsilon
\]
für alle $m,n> N$ gilt.
\end{definition}

Die Dezimalbruchapproximation von $\pi$ ist eine Cauchyfolge.
Sei $a_n$ der endliche Dezimalbruch mit $n$ Nachkommastellen, die
mit der Dezimalbruchentwicklung übereinstimmen.
Dann ist
$a_n-a_m$ ein Dezimalbruch, dessen erste $\min(n,m)$ Nachkommastellen
$0$ sind.
Insbesondere ist
\[
|a_n-a_m| < 10^{-\min(n,m)}.
\]
Sei also $\varepsilon>0$ gegeben und $N\in\mathbb{N}$ derart,
dass $10^{-N}<\varepsilon$.
Falls $n,m>N$ sind, folgt
\[
|a_n-a_m|
<
10^{-\min(n,m)}
<
10^{-N}
<
\varepsilon.
\]
Damit ist gezeigt, dass $(a_n)_{n\in\mathbb{N}}$ eine Cauchy-Folge ist.

Periodische Dezimalbrüche konvergieren gegen rationale Zahlen im Sinne
der folgenden Definition.

\begin{definition}[konvergente Folge]
Eine Cauchy-Folge $(a_n)_{n\in\mathbb{N}}$ konvergiert gegen $a\in\mathbb{Q}$,
auch geschrieben als
\[
\lim_{n\to\infty} a_n = a
\qquad\text{oder}\qquad
a_n\to a,
\]
wenn für jedes $\varepsilon>0$ ein $N\in\mathbb{N}$ existiert derart, dass
\[
|a_n-a|<\varepsilon
\]
für $n>N$.
\end{definition}

Nicht jede Cauchy-Folge konvergiert gegen eine rationale Zahl.
Die Approximationsfolge von $\!\sqrt{2}$ ist eine Cauchy-Folge, die nicht
konvergiert.

Mit Cauchy-Folgen kann man auch Rechnen.
In der Analysis wird der folgende Satz bewiesen.

\begin{satz}
Seien $(a_n)_{n\in\mathbb{N}}$ und $(b_n)_{n\in\mathbb{N}}$ Cauchy-Folgen.
Dann sind $(a_n+b_n)_{n\in\mathbb{N}}$, $(a_n-b_n)_{n\in\mathbb{N}}$ 
und $(a_nb_n)_{n\in\mathbb{N}}$ Cauchy-Folgen.
Falls $a_n\to a$ und $b_n\to b$, dann gilt auch
$a_n\pm b_n\to a\pm b$ und $a_nb_n\to ab$.
Falls $b\ne 0$ ist, ist auch $(a_n/b_n)_{n\in\mathbb{N}}$ ein Cauchy-Folge
und $a_n/b_n \to a/b$.
\end{satz}

\begin{definition}[Nullfolge]
Eine Cauchy-Folge $(a_n)_{n\in\mathbb{N}}$ heisst \emph{Nullfolge}, wenn
sie gegen $0$ konvergiert.
\index{Nullfolge}%
\end{definition}

Die Rechenoperationen und der Begriff einer Nullfolge ermöglichen jetzt,
Cauchy-Folgen miteinander zu identifizieren, die den gleichen Grenzwert
haben, selbst wenn dieser nicht eine rationale Zahl ist.

\begin{definition}
Zwei Cauchy-Folgen
$(a_n)_{n\in\mathbb{N}}$
und
$(b_n)_{n\in\mathbb{N}}$
heissen \emph{äquivalent}, in Zeichen $(a_n)\sim (b_n)$ wenn
$(a_n-b_n)_{n\in\mathbb{N}}$ eine Nullfolge ist.
\index{aquivalente@äquivalente Cauchy-Folgen}%
\end{definition}

Wenn zwei Cauchy-Folgen äquivalent sind, dann approximieren sie 
die gleiche Zahl.
Die gesuchte Erweiterung der Menge der rationalen Zahlen $\mathbb{Q}$
besteht also aus Cauchy-Folgen von rationalen Zahlen, wobei jedoch
äquivalente Folgen die gleiche Zahl repräsentieren.

\begin{definition}[Äquivalenzklassen]
Eine \emph{Äquivalenzklasse} $X$ der Äquivalenzrelation $\sim$ von
Cauchy-Folgen ist eine Menge von äquivalenten Cauchy-Folgen.
\end{definition}

Zwei Cauchy-Folgen
$(a_n)_{n\in\mathbb{N}}\in X$ 
und
$(b_n)_{n\in\mathbb{N}}\in X$ 
in einer Äquivalenzklasse $X$ von rationalen Cauchy-Folgen unterscheiden
sich nur durch eine Nullfolge, es gilt also
\[
\lim_{n\to\infty} (a_n-b_n) = 0.
\]
Die reellen Zahlen können jetzt als Äquivalenzklassen von Cauchy-Folgen
definiert werden.

\begin{definition}[Reelle Zahlen]
Die Menge der reellen Zahlen $\mathbb{R}$ ist die Menge der Äquivalenzklassen
von rationalen Cauchy-Folgen bezüglich der Äquivalenzrelation $\sim$.
\end{definition}

Diese Konstruktion zeigt, dass man mit reellen Zahlen rechnen kann, indem
man geeignete Approximationen dafür verwendet.
Welche Approximation verwendet wird, ist egal.
Wenn die Approximationsfolgen äquivalent sind, unterscheiden sich auch
Resultate von Rechnungen mit den Folgen durch Nullfolgen.
Die reellen Zahlen enthalten nach Konstruktionen alle Quadratwurzeln
und die Zahlen $e$ und $\pi$.
Jede Gleichung, für deren Lösung eine Approximationsfolge existiert,
kann daher in $\mathbb{R}$ auch gelöst werden.
So kann man zum Beispiel die Lösung der Gleichung
\[
y = 1 + \frac1y
\]
dadurch approximieren, dass man die Folge $(y_n)_{n\in\mathbb{N}}$
durch
\[
y_0 = 1,
\quad
y_{n+1} = 1 + \frac1{y_n}
\]
bildet.
$(y_n)_{n\in\mathbb{N}}$ ist eine Cauchy-Folge, die gegen das Verhältnis
$\varphi=(\!\sqrt{5}+1)/2$ des goldenen Schnittes konvergiert (siehe
Abbildung~\ref{buch:komplex:zahlen:table:gold}, zweite Spalte).
%
% table-gold.tex
%
% (c) 2026 Prof Dr Andreas Müller
%
\begin{table}
\centering
\begin{tabular}{|>{$}r<{$}>{$}l<{$}|>{$}r<{$}|}
\hline
n & y_n & x_n \\
\hline
 1 & 2                 & 0.508034707497476
\\
 2 & 1.500000000000000             & -0.730167373438399 \\
 3 & 1.\underline{6}66666666666667 &  0.319690815934174 \\
 4 & 1.\underline{6}00000000000000 & -1.404165739925735 \\
 5 & 1.\underline{6}25000000000000 & -0.345999548896763 \\
 6 & 1.\underline{61}5384615384615 &  1.272088814812140 \\
 7 & 1.\underline{61}9047619047619 &  0.242990090617789 \\
 8 & 1.\underline{61}7647058823529 & -1.936202034966178 \\
 9 & 1.\underline{618}181818181818 & -0.709863503540628 \\
10 & 1.\underline{61}7977528089888 &  0.349429012666962 \\
11 & 1.\underline{6180}55555555556 & -1.256191291052456 \\
12 & 1.\underline{6180}25751072961 & -0.230067094013908 \\
13 & 1.\underline{61803}7135278515 &  2.058245522488201 \\
14 & 1.\underline{61803}2786885246 &  0.786197418015100 \\
15 & 1.\underline{61803}4447821682 & -0.242873870579829 \\
16 & 1.\underline{618033}813400125 &  1.937244794475075 \\
17 & 1.\underline{61803}4055727554 &  0.710523884635529 \\
18 & 1.\underline{6180339}63166706 & -0.348444169203709 \\
19 & 1.\underline{6180339}98521804 &  1.260728028475482 \\
20 & 1.\underline{61803398}5017358 &  0.233767770871423 \\
21 & 1.\underline{6180339}90175597 & -2.021990939507587 \\
22 & 1.\underline{618033988}205325 & -0.763714440828034 \\
23 & 1.\underline{618033988}957902 &  0.272837745754085 \\
24 & 1.\underline{618033988}670443 & -1.696172136911838 \\
25 & 1.\underline{6180339887}80243 & -0.553304666781480 \\
26 & 1.\underline{6180339887}38303 &  0.627008940439556 \\
27 & 1.\underline{6180339887}54322 & -0.483932324938967 \\
28 & 1.\underline{61803398874}8204 &  0.791236155774208 \\
29 & 1.\underline{6180339887}50541 & -0.236304511027909 \\
30 & 1.\underline{618033988749}648 &  1.997761646531475 \\
\hline
\infty & 1.618033988749895 & \\
\hline
\end{tabular}
\caption{Approximation der Zahl $\varphi=(1+\!\sqrt{5})/2$
durch die Folge $(y_n)_{n\in\mathbb{N}}$, die durch $y_0=1$ und
$y_{n+1}=1+1/y_n$ rekursiv definiert ist (Spalte $y_n$).
In der Spalte $x_n$ sind die Newton-Approximationen für die
Funktion $f(x) = x^2 + 1$, rekursiv definiert durch
$x_{n+1} = x_n -  f(x_n)/f'(x_n)$, dargestellt, die keinerlei
Anzeichen von Konvergenz zeigen.
\label{buch:komplex:zahlen:table:gold}}
\end{table}


Approximationsfolgen für die Nullstellen einer Funktion $f(x)$ können
zum Beispiel durch das Newton-Verfahren gefunden werden.
Dazu wählt man eine Anfangsschätzung $x_0$ und bildet dann die 
Folge
\[
x_{n+1}
=
x_n - \frac{f(x_n)}{f'(x_n)}.
\]
Falls $x_0$ genügend nahe bei einer Nullstelle $\hat{x}$ liegt, wird die
Folge $(x_n)_{n\in\mathbb{N}}$ gegen die Nullstelle konvergieren:
\[
\lim_{n\to\infty} x_n
=
\hat{x}.
\]
Wendet man dies auf die Funktion $f(x)=x^2+1$ an, welche $f(x) \ge 1$
für alle $x\in\mathbb{R}$ keine reelle Nullstelle hat, erhält man eine
Folge, die keine Cauchy-Folge ist
(Abbildung~\ref{buch:komplex:zahlen:table:gold}, dritte Spalte).
Die Gleichung
\begin{equation}
x^2 + 1 = 0
\qquad\Leftrightarrow\qquad
x^2 = -1
\end{equation}
hat also keine Lösung, die mit einer Approximationsfolge gefunden werden
kann.

%
% Algebraische Erweiterungen von Q
%
\subsubsection{Algebraische Erweiterungen von $\mathbb{Q}$}
Die Vervollständigung der der rationalen Zahlen zu den reellen
Zahlen stellt sicher, dass jede Cauchy-Folge einen Grenzwert hat.
Sie führt aber zu einem neuen Problem, welches ein Hindernis für
Computeralgebrasysteme wird.
Da reelle Zahlen nur durch ihre Approximationen definiert sind,
lässt sich damit grundsätzlich nur dann exakt rechnen, wenn auch
noch eine alternative Charakterisierung zur Verfügung steht.
Mit der Kreiszahl $\pi$ kann man nur desshalb exakt rechnen, weil
sie die kleinste positive Nullstelle der Sinusfunktion ist.

Für die Quadratwurzel $\!\sqrt{2}$, die früher als erstes Beispiel
einer irrationalen Zahl in den Vordergrund gestellt wurde, gibt
es eine solche alternative Charakterisierung.
Beim algebraischen Rechnen mit $\!\sqrt{2}$ verwendet man fast
instinktiv die Beziehung $(\!\sqrt{2})^2=2$, zum Beispiel
beim Ausmultiplizieren von
\[
(\!\sqrt{2}+1)^2
=
(\!\sqrt{2})^2
+2\!\sqrt{2}\cdot 1 
+ 1^2
=
2 + 2\!\sqrt{2} + 1
=
3+2\!\sqrt{2}.
\]

Wir wollen diese Beobachtung noch etwas expliziter machen und
bezeichnen mit $\alpha$ ein Objekt, für welches sowohl die Regeln
der Algebra ebenso gelten wie die Identität $\alpha^2-2=0$.
Durch die Ersetzung $\alpha^2 = 2$ kann jede Potenz von $\alpha$
auf entweder ein Vielfaches von $\alpha$ oder auf eine ganze Zahl
reduziert werden, es gilt nämlich
\[
\alpha^n
=
\begin{cases}
2^{\frac{n}2}         &\qquad\text{$n$ gerade}\\
2^{\frac{n-1}2}\alpha &\qquad\text{$n$ ungerade.}
\end{cases}
\]
Für Linearkombinationen $a+b\alpha$ und $c+d\alpha$ lassen sich
jetzt einfache Rechenregeln aufstellen:
\begin{align*}
(a+b\alpha)\pm(c+d\alpha) &= (a\pm b) + (c\pm d)\alpha
\\
(a+b\alpha)(c+d\alpha) &= (ac + 2bd) + (ad+bc)\alpha.
\end{align*}
Als Spezialfall der Produktregel findet man
\[
(a+b\alpha)(a-b\alpha)
=
a^2 - 2 b^2,
\]
daher heisst $a-b\alpha$ auch das zu $a+b\alpha$ \emph{konjugierte}
Element.
Durch Erweitern mit $(c-d\alpha)$ kann man sogar den Kehrwert
\begin{align*}
\frac{1}{c+d\alpha}
&=
\frac{1}{c+d\alpha}
\cdot
\frac{c-d\alpha}{c-d\alpha}
=
\frac{c-d\alpha}{c^2-2d^2}
=
\frac{c}{c^2-2d^2} - \frac{d}{c^2-2d^2}\alpha
\intertext{und damit auch den Quotienten}
\frac{a+b\alpha}{c+d\alpha}
&=
\frac{ac-2bd}{c^2-2d^2} - \frac{ad-bc}{c^2-2d^2}\alpha
\end{align*}
definieren.

Alle vier Grundoperationen lassen sich also für alle Zahlen der Form 
$a+b\alpha$ ausschliesslich durch Verwendung der arithmetischen Gesetze
und der Identität $\alpha^2-2$ definieren und sie ergeben immer wieder
Zahlen dieser Form.
Es entsteht somit eine neue Menge
\[
\mathbb{Q}(\alpha)
=
\{a+b\alpha\mid a,b\in\mathbb{Q}\},
\]
in der die gleichen Rechengesetze gelten wie in $\mathbb{Q}$.
Die Menge $\mathbb{Q}$ ist eine echte Teilmenge, $\mathbb{Q}(\alpha)$
ist also eine echte Erweiterung von $\mathbb{Q}$.
In $\mathbb{Q}(\alpha)$ lässt sich jetzt die Gleichung $x^2-2=0$ lösen.
Es ist nämlich
\[
(x+\alpha)(x-\alpha)
=
x^2-\alpha^2
=
x^2 - 2
=
0,
\]
die beiden Lösungen der Gleichung $x^2-2=0$ sind also $\pm\alpha$.

Die Konstruktion lässt sich noch etwas verallgemeinern, indem das
Polynom $x^2-2$ durch ein beliebiges normiertes Polynom $m(x)$ aus der
Menge
\[
\mathbb{Q}[x]
=
\{
x^n+m_{n-1}x^{n-1}+\dots+m_1x+m_0 
\mid
m_0,\dots,m_{n-1}\in\mathbb{Q}
\}
\]
der Polynome mit rationalen Koeffizienten ersetzt wird.
Wieder versucht man ein neues Objekt $\alpha$ durch die Eigenschaft
$m(\alpha)=0$ zu charakterisieren.
Dies führt jedoch auf Schwierigkeiten, wenn sich $m$ als Produkt
$m=ab$ von Polynomen $a,b\in\mathbb{Q}[x]$ schreiben lässt, denn die beiden
Faktoren $a(x)$ und $b(x)$ könnten Nullstellen mit widersprüchlichen
Eigenschaften definieren.

\begin{definition}[irreduzibles Polynom]
Das Polynom $m\in\mathbb{Q}[x]$ heisst \emph{irreduzibel}, wenn es keine
\index{irreduzibel}%
Polynome vom Grad $>0$ gibt, die $m$ als Produkt haben.
\end{definition}

Das Polynom $x^2-2$ ist irreduzibel in $\mathbb{Q}[x]$, es lässt sich
nicht in zwei Faktoren zerlegen.
Die einzig mögliche Faktorisierung wäre $x^2-2=(x+\!\sqrt{2})(x-\!\sqrt{2})$,
aber die Faktoren $x+\!\sqrt{2}$ und $x-\!\sqrt{2}$ haben nicht
alle Koeffizienten in $\mathbb{Q}$.

In einem irreduziblen Polynom $m$ kann der Koeffizient $m_0$ nicht
verschwinden, denn dann liesse sich
\[
m
=
(x^{n-1} + m_{n-1}x^{n-2}+\dots + m_1)x
\]
als Produkt zweier Polynome vom Grad $n-1$ bzw.~1 schreiben.

Mit einem irreduziblen Polynom $m$ definieren wir jetzt wieder das Objekt
$\alpha$ mit der Eigenschaft $m(\alpha)=0$.
$m$ heisst auch das \emph{Minimalpolynom} von $\alpha$.
\index{Minimalpolynom}%
Ausgeschrieben bedeutet 
\[
m(\alpha)
=
\alpha^n
+
m_{n-1}\alpha^{n-1}
+
\dots
+
m_1\alpha
+
m_0
=
0
\quad\Rightarrow\quad
\alpha^n
=
-m_{n-1}\alpha^{n-1}
-
\dots
-
m_1\alpha
-
m_0.
\]
Man kann also jede Potenz grösser oder gleich $n$ von $\alpha$ durch
eine Linearkombination von Potenzen $\alpha^k$ mit $0\le k< n$ 
ausdrücken.

\begin{definition}
Sei $m\in\mathbb{Q}[x]$ ein irreduzibles Polynom vom Grad $n$ der Form
\[
m(x)
=
x^n + m_{n-1}x^{n-1} + \dots + m_1x + m_0.
\]
Die Menge
\[
\mathbb{Q}(\alpha)
=
\{
a_0 + a_1\alpha + \dots + a_{n-1}\alpha^{n-1}
\mid
a_k\in\mathbb{Q}\text{ mit } 0\le k<n
\}
\]
heisst die \emph{Erweiterung} von $\mathbb{Q}$ um $\alpha$.
\end{definition}

Wieder sollen die Rechengesetze durch Übertragung der gewohnten
Regeln entstehen, jedoch müssen alle höheren Potenzen von $\alpha$
jeweils wieder auf Potenzen $\alpha^k$, $0\le k<n$, reduziert werden.
Da der Koeffizient $m_0$ des irreduziblen Polynoms $m$ nicht verschwindet,
kann man die Gleichung $m(\alpha) = 0$ mit $\alpha^{-1}$ multiplizieren
und nach $\alpha^{-1}$ auflösen:
\begin{align*}
m(\alpha)\cdot \alpha^{-1}
=
0
&=
\alpha^{n-1}
+
m_{n-1}\alpha^{n-2}
+
\dots
+
m_1
+
m_0\alpha^{-1}
\\
\alpha^{-1}
&=
-\frac{1}{m_0}
(
\alpha^{n-1}
+
m_{n-1}\alpha^{n-2}
+
\dots
+
m_1
)
\\
&=
-\frac{m_1}{m_0}
-\dots
-\frac{m_{n-1}}{m_0}\alpha^{n-2}
-\frac{1}{m_0}\alpha^{n-1}
\in
\mathbb{Q}(\alpha).
\end{align*}
Das reziproke Element ist also immer auch in $\mathbb{Q}(\alpha)$.

\begin{beispiel}
\label{buch:komplex:zahlen:bsp:kubikwurzel}
Die Gleichung $x^3-2=0$ lässt sich in $\mathbb{Q}$ nicht lösen.
Das Polynom $m(x)=x^3-2$ ist irreduzibel als Polynom in $\mathbb{Q}[x]$.
Die Erweiterung $\mathbb{Q}(\alpha)$ um das Objekt $\alpha$ mit
der Eigenschaft $\alpha^3-2=0$ ist also die Menge
\[
\mathbb{Q}(\alpha)
=
\{
a_0+a_1\alpha + a_2\alpha^2
\mid
a_0,a_1,a_2\in\mathbb{Q}
\}.
\]
Die Additionsregeln sind nicht weiter anspruchsvoll, wir berechnen
daher nur das Produkt von zwei Elementen $a,b\in\mathbb{Q}(\alpha)$:
\begin{align*}
ab
&=
(a_0+a_1\alpha+a_2\alpha^2)
(b_0+b_1\alpha+b_2\alpha^2)
\\
&=
a_0b_0
+
(a_0b_1+a_1b_0)\alpha
+
(a_0b_2+a_1b_1+a_2b_0)\alpha^2
+
(a_1b_2+a_2b_1){\color{darkred}\alpha^3}
+
a_2b_2{\color{darkred}\alpha^4}.
\intertext{Die Potenzen $\alpha^3$ und $\alpha^4$ müssen durch Anwendung der
Reduktionsregel $\alpha^3=2$ auf}
&=
a_0b_0
+
(a_0b_1+a_1b_0)\alpha
+
(a_0b_2+a_1b_1+a_2b_0)\alpha^2
+
(a_1b_2+a_2b_1){\color{darkred}\cdot 2}
+
a_2b_2{\color{darkred} \cdot 2\alpha}
\intertext{reduziert werden.
Das Produkt ist daher}
ab
&=
(a_0b_0 + 2a_1b_2+2a_2b_1)
+
(a_0b_1+a_1b_0 + 2a_2b_2)\alpha
+
(a_0b_2+a_1b_1+a_2b_0)\alpha^2
\in\mathbb{Q}(\alpha).
\end{align*}
Das zu $\alpha$ reziproke Element ist
\[
\alpha^{-1}
=
\frac{1}{2}\alpha^2,
\]
tatsächlich ist
\[
\alpha\cdot \alpha^{-1}
=
\alpha
\cdot
\frac{1}{2}\alpha^2
=
\frac{\alpha^3}{2}
=
\frac{2}{2}
=
1.
\]
In $\mathbb{Q}(\alpha)$ kann man jetzt auch das Polynom $x^3-2$ 
faktorisieren, indem man durch den Faktor $(x-\alpha)$ dividiert.
Die Polynomdivision ergibt
\[
\renewcommand{\arraycolsep}{2.5pt}
\begin{array}{rcrcrcrcrcrcrcrcr}
\llap{$($}
x^3 & &           & &            &-&        2\rlap{$)$}&\phantom{(}:\phantom{)}&\llap{$($} x &-& \alpha\rlap{$)$} &\phantom{(}=\phantom{)}& x^2 &+& \alpha x &+& \alpha^2 \\
x^3 &-&\alpha x^2 & &            & &         & &   & &        & &     & &          & &          \\
\cline{1-7}
    & &\alpha x^2 & &            &-&        2& &   & &        & &     & &          & &          \\
    & &\alpha x^2 &-& \alpha^2 x & &         & &   & &        & &     & &          & &          \\
\cline{3-7}
    & &           & & \alpha^2 x &-&        2& &   & &        & &     & &          & &          \\
    & &           & & \alpha^2 x &-& \alpha^3& &   & &        & &     & &          & &          \\
\cline{5-7}
    & &           & &            & & \alpha^3\rlap{$\mathstrut-2 = 0$.}& &   & &        & &     & &          & &
\end{array}
\]
Tatsächlich ergibt Ausmultiplizieren zur Kontrolle
\begin{align}
(x-\alpha)
(x^2 + \alpha x + \alpha^2)
&=
x^3 +\alpha x^2 +\alpha^2 x
-
\alpha
x^2
-
\alpha^2 x
-
\alpha^3
\label{buch:komplex:zahlen:eqn:faktorisierung}
\\
&=
x^3-\alpha^3
\notag
\\
&=
x^3-2.
\notag
\end{align}
In der üblichen algebraischen Notation würde man $\alpha=\sqrt[3]{2}$
für die Kubikwurzel schreiben.

Wir erwarten aber, dass die Gleichung $x^3-2=0$ zwei weitere Lösungen
hat, die sich aus der Faktorisierung
\eqref{buch:komplex:zahlen:eqn:faktorisierung}
berechnen lassen sollten.
Das Polynom $x^2 +\alpha x + \alpha^2$ ist allerdings irreduzibel
als Polynom in $\mathbb{Q}(\alpha)[x]$, es definiert also eine neue
Erweiterung um eine Zahl $\beta$, die das Minimalpolynom
$b(x)=x^2+\alpha x + \alpha^2$ hat.
Die Faktorisierung ist also erst in $\mathbb{Q}(\alpha)(\beta)
=\mathbb{Q}(\alpha,\beta)$ möglich.

Die konventionelle Schreibweise für die Nullstelle des Polynoms
$b$ ergibt sich aus der Lösungsformel für quadratische Gleichungen.
Man muss
\begin{equation}
\beta
=
-\frac{\alpha}{2}
+
\!\sqrt{\rlap{$\phantom{\bigg|}$}\smash{\biggl(\frac{\alpha}{2}\biggr)^2-\alpha^2}}
=
-\frac{\alpha}{2}
+
\!\sqrt{-\frac34\alpha^2}
\label{buch:komplex:zahlen:eqn:cbrtnegativ}
\end{equation}
setzen.
\end{beispiel}

Das Beispiel zeigt, wie die Lösung einer Gleichung möglicherweise
das wiederholte Hinzufügen neuer Zahlen nötig macht.
Die Notation $\sqrt[3]{2}$ für das Element $\alpha$ ist gefährlich,
da sie bereits suggeriert, dass $\alpha$ eine reelle Zahl sein
soll.
Wir haben aber nur gefordert, dass $\alpha$ eine beliebige
Nullstelle des Polynoms sein soll.
Die Darstellung \eqref{buch:komplex:zahlen:eqn:cbrtnegativ} der
anderen Nullstellen zeigt für $\alpha=\sqrt[3]{2}$, dass diese
wegen
\[
%-\frac{\sqrt[3]{2}}2
%\pm
%\!\sqrt{
%-\frac34(
%\sqrt[3]{2}
%)^2
%}
%\Rightarrow
-\frac34 \sqrt[3]{2} < 0
\]
keine reellen Zahlen sein können.
Dieses Argument verwendet aber die Ordnungsrelation in den reellen
Zahlen und insbesondere die Eigenschaft, dass negative Zahlen
keine reellen Quadratwurzeln haben.
Auf die Ordnungsreqlation wurde in der Konstruktion von
$\mathbb{Q}(\alpha)$ nirgends Bezug genommen.
Sogar die Konstruktion von $\mathbb{Q}(\!\sqrt{2})$ hätte genauso 
gut im $-\!\sqrt{2}$ an Stelle von $\alpha$ funktionieren können.
Mit rein arithmetischen Mitteln sind die beiden Nullstellen von
$x^2-2$ nicht unterscheidbar.

\begin{definition}[algebraisches Element]
Ein Element $\alpha$ heisst \emph{algebraisch über $\mathbb{Q}$},
wenn es ein Polynom $m\in\mathbb{Q}[x]$ gibt, welches $m(\alpha=0$
erfüllt.
\index{algebraisch}%
\end{definition}

Die Definition von $\sqrt{2}$ oder $\sqrt[3]{2}$ als
algebraische Elemente über $\mathbb{Q}$ ermöglicht also,
exakt mit diesen Elementen zu rechnen, ohne dass auf
Approximationen zurückgegriffen werden muss.
Die algebraische Definition ist daher die Methode der
Wahl für ein Computeralgebrasystem.

Da die Menge aller rationalen Polynome abzählbar unendlich ist,
kann es nur abzählbar viele über $\mathbb{Q}$ algebraische 
Zahlen geben.
Die Menge $\mathbb{R}$ der reellen Zahlen ist aber überabzählbar
unendlich, so dass die meisten reellen Zahlen nicht algebraisch
sind.
Sie heissen transzendent.

\begin{definition}[transzendent]
Ein Element $\alpha$ heisst \emph{transzendent} über $\mathbb{Q}$,
wenn es nicht algebraisch ist.
\end{definition}

Die Zahlen $\pi$ und $e$ sind beide transzendent.
Der Nachweis, dass eine Zahl transzendent ist, ist oft schwierig.
Schon 1844 hat Jospeh Liouville vermutet, dass $e$ transzendent 
ist, der Nachweis ist aber Charles Hermite erst 1873 gelungen
\cite{buch:hermitebeweis}.
Für $\pi$ ist der Beweis deutlich anspruchsvoller.
Nach einem Satz von Lindemann und Weierstrass
\cite{buch:lindemann-weierstrass} sind für verschiedene
algebraische Zahlen $\alpha_1,\dots,\alpha_n$
die Zahlen $e^{\alpha_1},\dots,e^{\alpha_n}$ linear unabhängig
über $\mathbb{Q}$.
Die Zahl $0$ ist sicher algebraisch.
Wäre auch $\pi$ eine algebraische Zahl, dann wäre auch $i\pi$ algebraisch.
Nach dem Satz von Lindemann-Weierstrass müssten dann $e^0=1$ und
$e^{i\pi}=-1$ über den algebraischen Zahlen linear unabhängig sein,
was sie wegen der eulerschen Identität
\[
e^{i\pi}+1=0
\]
offensichtlich nicht sind.
Der Widerspruch zeigt, dass $\pi$ transzendent sein muss.

%
% Algebraische Erweiterungen von F_p
%
\subsubsection{Algebraische Erweiterungen von $\mathbb{F}_p$}
Die Konstruktion der reellen Zahlen ist nur möglich, weil es auf
der Menge $\mathbb{Q}$ der rationalen Zahlen eine Ordnungsrelation
gibt, die der Idee des Abstands $|a_n-a_m|$ von Folgengliedern
einen Sinn gibt.
Die Definition eines algebraischen Elementes hingegen ist auch dann 
möglich, wenn es keine solche Ordnungsrelation gibt.
Die endlichen Körper $\mathbb{F}_p$ für Primzahlen $p$ sind von
dieser Art.

\begin{definition}[endliche Körper $\mathbb{F}_p$]
\index{endliche Körper}%
\index{Korper@Körper!endlich}%
Sei $p$ eine Primzahl.
Die Menge
\[
\mathbb{F}_p
=
\{ 0, \dots , p-1 \}
\]
der ganzzahligen Reste bei Division durch $p$
heisst der \emph{endlich Körper} der Ordnung $p$.
\index{Fp@$\mathbb{F}_p$}%
\end{definition}

In $\mathbb{F}_p$ sind wie in $\mathbb{Q}$ alle vier Grundoperationen
wohldefiniert, insbesondere auch das reziproke Element.
Diese letzte Aussage ist gleichbedeutend damit, dass die Multiplikation
\[
x\mapsto a\cdot x
\]
mit einem Element $a\in\mathbb{F}_p\setminus\{0\}$ eine Bijektion ist.
Da dies eine lineare Abbildung ist, muss man nur prüfen, ob es ein
Element $x\ne 0$ gibt derart, dass $ax=0$ in $\mathbb{F}_p$ ist.
Dies ist gleichbedeutend damit, dass 
\[
a\cdot n \equiv 0 \mod p
\qquad\Leftrightarrow\qquad
p | an
\]
gilt.
Da weder $a$ noch $n$ von $p$ geteilt werden können und $p$ eine
Primzahl ist, kann es eine solche Zahl $n$ nicht geben.

Da alle ganzen Zahlen, die sich nur um ein Vielfaches von $p$
unterscheiden, das gleiche Element in $\mathbb{F}_p$ darstellen,
kann es keine mit den arithmetischen Operationen verträgelich
Ordnungsrelation geben.
Schon die Unterschiedung zwischen positiven und negativen
Zahlen ist sinnlos, da jedes Element $a$ mit $0\le a<p$ auch
eine Darstellung $a-p$ hat, welche negativ ist.
Es folgt, dass es auch nicht sinnvoll sein kann, in $\mathbb{F}_p$
von Folgen und Grenzwerten zu sprechen.
So etwas wie eine Vervollständigung von $\mathbb{F}_p$ wie
die Vervollständigung von $\mathbb{Q}$ zu $\mathbb{R}$ kann es
also nicht geben.

Trotzdem ist es sinnvoll, von algebraischen Erweiturungen zu  sprechen.
Wir betrachten dazu statt Polynome über $\mathbb{Q}$ solche über
$\mathbb{F}_p$.
Wieder setzen wir
\[
\mathbb{F}_p[x]
=
\{
x^n + a_{n-1}x^{n-1} + \dots + a_1x + a_0
\mid
a_0,\dots,a_{n-1}\in \mathbb{F}_p
\}.
\]
Wie im Falle von rationalen Koeffizienten ist ein irreduzibles
Polynom eines, welches sich nicht als Produkt zweier Polynome von
positivem Grad schreiben lässt.
Dabei kann es aber durchaus passieren, dass ein Polynom, welches
über $\mathbb{Q}$ irreduzibel war, als Polynom über $\mathbb{F}_p$
betrachtet nicht mehr irreduzibel sein muss.

\begin{beispiel}
Für $p=5$ ist
\[
(x+3)(x+2)
=
x^2 + (3+2)x + 6
=
x^2 + 1
\in
\mathbb{F}_5[x].
\]
Das Polynom $x^2+1\in\mathbb{F}_5[x]$ ist also nicht irreduzibel,
im Gegensatz zum Polynom $x^2+1\in\mathbb{Q}[x]$, welches tatsächlich
irreduzibel ist.
Dies zeigt auch, dass im Gegensatz zu $\mathbb{Q}$ oder $\mathbb{R}$,
die Gleichung $x^2+1=0$ in $\mathbb{F}_p$ eine Lösung hat.
\end{beispiel}

Wie für rationale Koeffizienten lassen sich auch für ein algebraisches
Element über $\mathbb{F}_p$ die arithmetischen Grundoperationen
definieren.

\begin{beispiel}
\label{buch:komplex:zahlen:bsp:F5}
Für $p=5$ ist $x^2-2=x^2+3\in\mathbb{F}_5[x]$ ein irreduzibles Polynom,
wie man durch Durchprobieren der möglichen Faktoren $(x-a)(x-b)$ mit
$a,b\in\mathbb{F}_5$ nachprüfen kann.
Sei jetzt $\alpha$ ein Objekt, welches $\alpha^2=2$ erfüllt.
Die Rechenoperationen sind ganz analog wie im Falle rationaler
Koeffizienten definiert.
\end{beispiel}

Die Nullstelle $\alpha$ des Polynoms $x^2-2\in\mathbb{F}_4[x]$,
die in Beispiel~\ref{buch:komplex:zahlen:bsp:F5} zu $\mathbb{F}_5$
hinzugefügt worden ist, ist eine Quadratwurzel von $2$ in $\mathbb{F}_5$,
hat aber nichts mit der vertrauten Quadratwurzeln
$\!\sqrt{5}\approx 1.414213562373$ in $\mathbb{R}$ zu tun.
Während es für reelle Nullstellen von Polynomen in $\mathbb{Q}[x]$ 
immer eine Dezimaldarstellung gibt, gibt es keine alternative Darstellung
für $\alpha$.

\begin{beispiel}
Das Polynom $m(x)=x^3-2=x^3+5\in\mathbb{F}_7[x]$ ist irreduzibel, denn
wenn es faktorisiert werden könnte, müsste notwendigerweise einer der
Faktoren den Grad 1 haben.
Ein solcher Faktor wäre also von der Form $x-a$, wobei $a\in\mathbb{F}_7$
sein müsste.
Das Polynom müsste also eine Nullstelle in $\mathbb{F}_7[x]$ haben.
Die dritten Potenzen aller Elemente von $\mathbb{F}_7$ sind allerdings
\[
\begin{tabular}{|>{$}l<{$}|>{$}c<{$}>{$}c<{$}>{$}c<{$}>{$}c<{$}>{$}c<{$}>{$}c<{$}>{$}c<{$}|}
\hline
a   & 0 & 1 & 2 & 3 & 4 & 5 & 6 \\
\hline
a^3 & 0 & 1 & 1 & 6 & 1 & 1 & 6 \\
\hline
\end{tabular}
\]
Keine dritte Potenz ergibt 2.

Fügt man $\mathbb{F}_7$ das Element $\alpha$, welches eine Nullstelle
von $m$ ist, zu $\mathbb{F}_7$ hinzu, erhält man neu die Menge
$\mathbb{F}_7(\alpha)$.
Lässt man Koeffizienten aus $\mathbb{F}_7(\alpha)$ zu, lässt sich das
Polynom $m$ wie in Beispiel~\ref{buch:komplex:zahlen:bsp:kubikwurzel}
als
\[
x^3-2
=
(x-\alpha)(\underbrace{x^2+\alpha x + \alpha^2}_{\displaystyle = b})
\]
faktorisieren.
Auch in diesem Beispiel ist eine weitere Erweiterung
$\mathbb{F}_7(\alpha,\beta)$ von $\mathbb{F}_p(\alpha)$ durch eine
Nullstelle $\beta$ des Polynoms $b\in\mathbb{F}_7(\alpha)$ nötig,
damit die anderen zwei Nullstellen von $m$ ebenfalls gefunden werden
können.
\end{beispiel}

Die Adjunktion einer Nullstelle eines irreduziblen Polynoms 
aus $\mathbb{F}_p[x]$ zu $\mathbb{F}_p$ erzeugt also wieder
die Menge 
\[
\mathbb{F}_p(\alpha)
=
\{
a_0+a_1\alpha + \dots a_{n-1}\alpha^{n-1}
\mid
a_0,\dots,a_{n-1}\in\mathbb{F}_p
\},
\]
die ein Vektorraum über $\mathbb{F}_p$ ist, mit den $n$ Monomen
\[
1, \alpha, \dots, \alpha^{n-1}
\]
als Basis.
$\mathbb{F}_p(\alpha)$ enthält somit $p^n$ Elemente.

\begin{definition}[endliche Körper]
Die Menge $\mathbb{F}_p(\alpha)$ mit $p^n$ Elementen heisst der
\emph{endliche Körper mit $p^n$} Elementen.
\index{endliche Körper}%
\index{Korper@Körper!endlich}%
\end{definition}

Fügt man einem Körper mit $p^n$ Elementen ein weiteres Element mit
einem Minimalpolynom vom Grad $l$ hinzu, entsteht eine Menge mit
mit $(p^n)^l=p^{nl}$ Elementen.
Man kann zeigen, dass alle Körper mit $p^n$ wie man sagt isomorph sind,
ganz unabhängig davon, was für ein Minimalpolynom verwendet wird
und schreibt daher unabhängig vom Minimalpolynom
$\mathbb{F}_p(\alpha)=\mathbb{F}_{p^n}$.
Die endlichen Körper spielen eine bedeutende Rolle in der Kryptographie.
Zum Beispiel verwendet die Spezifikation des AES-Algorithmus die 
Arithmetik in einem Körper $\mathbb{F}_{2^8}$ mit $2^8$ Elementen,
die im Code als Bytes dargestellt werden.

%
% Die ganzen gaussschen Zahlen Z[i]
%
\subsubsection{Die ganzen gaussschen Zahlen $\mathbb{Z}[i]$}
Die Erweiterung einer Zahlenmenge um ein neues Element, welches
als Nullstelle eines irreduziblen Polynoms, lässt sich auch auf
die Menge $\mathbb{Z}$ anwenden.
Für das Polynom $m = x^2+1\in\mathbb{Z}[x]$ wird die Nullstelle
mit $i$ bezeichnet.
Die Rechenregeln folgen unmittelbar aus den Rechenregeln für
Polynome mit ganzzahligen Koeffizienten durch Anwendung der
Identität $i^2+1=0$ oder $i^2=-1$.
Es entsteht die Menge der ganzen gaussschen Zahlen gemäss der
folgenden Definition.

\begin{definition}[ganze gausssche Zahlen]
Die Elemente
\[
\mathbb{Z}[i]
=
\{
a+bi
\mid
a,b\in\mathbb{Z}
\}
\]
heisst die Menge der ganzen gaussschen Zahlen.
Sie bildet einen kommutativen Ring mit Eins, der den Ring $\mathbb{Z}$
als echten Unterring enthält.
\end{definition}

Die Ringoperationen sind
\begin{align*}
\text{Addition:}&&
(a+bi)\pm(c+di)
&=
(a\pm c) + (b\pm d)i
\\
\text{Multiplikation:}&&
(a+bi)(c+di)
&=
ac-bd
+
(ad+bc)i.
\end{align*}
Nicht jede ganze gausssche Zahl lässt sich invertieren.
Die Inverse wird durch Erweiterung mit $a-bi$ als
\[
\frac{1}{a+bi}\cdot\frac{a-bi}{a-bi}
=
\frac{a-bi}{a^2+b^2}
=
\frac{a}{a^2+b^2}
-
\frac{b}{a^2+b^2}i
\]
gefunden.
Die Inverse existert also genau dann, wenn $a^2+b^2=1$ ist.
Dies ist nur für die Zahlen $\pm 1$ und $\pm i$ der Falle.

\begin{definition}[Betrag]
Für eine ganze gausssche Zahl $z=a+bi\in\mathbb{Z}[i]$ heisst die
Zahl $|z|$ definiert durch $|z|^2 = a^2+b^2$ der \emph{Betrag} von $z$.
\index{Betrag}%
\end{definition}

Für zwei das Produkt $z=(a+bi)(c+di)$ zweier ganzer gaussscher Zahlen gilt
\begin{align*}
|z|^2
&=
(ac-bd)^2 + (ad+bc)^2
\\
&=
a^2c^2-2abcd+b^2d^2
+
a^2d^2
+
2abcd
+
b^2c^2
\\
&=
a^2c^2 + a^2d^2 + b^2c^2 + b^2d^2
\\
&=
(a^2+b^2)(c^2+d^2)
\\
&=
|a+bi|^2 |c+di|^2.
\end{align*}
Mit anderen Worten: Wenn sich eine ganze gausssche Zahl faktorisieren
lässt, dann lassen sich auch die Beträge faktorisieren.

Tatsächlich kann man im Ring $\mathbb{Z}[i]$ der ganzen gausschen Zahlen
eine Theorie der Faktorisierung aufstellen, die ganz analog ist zur
Theorie der Primfaktoren in $\mathbb{Z}$.

%
% Der Körper der komplexenzahlen
%
\subsection{Der Körper der komplexen Zahlen}
In der bisherigen Entwicklung haben wir uns vor allem um einzelne
Zahlen gekümmert.
Für die Menge der Zahlen als ganzes und die darauf definierten 
Rechenoperationen haben wir uns nur indirekt gekümmert.

%
% Was ist ein Körper
%
\subsubsection{Was ist ein Körper?}
Die moderne Mathematik hat die möglichen Strukturen von Zahlenmengen und
ihren Operationen im Detail studiert und entsprechend ihren Eigenschaften
klassifiziert.
Die einfachste Art einer Struktur geht von nur einer Operation aus.
Die zugehörige Struktur ist die einer Gruppe.

\begin{definition}[Gruppe]
Eine \emph{Gruppe} ist eine Menge $G$ mit einer zweistelligen Operation
\[
\cdot
:
G\times G \to G
:
(g,h) \mapsto gh
\]
mit den folgenden Eigenschaften.
\begin{enumerate}
\item
Die Operation ist \emph{assoziativ}: $(gh)l = g(hl)$ für alle $g,h,l\in G$.
\index{assoziativ}%
\item
Es gibt ein \emph{neutrales Element} $e\in G$, es gilt $ea=a$ für alle $a\in G$.
\index{neutrales Element}%
\index{Element!neutral}%
\item
Zu jedem Element $g\in G$ gibt es ein inverses Element $g^{-1}\in G$
mit $g^{-1}g=e$.
\end{enumerate}
Die Gruppe heisst \emph{kommutativ}
\index{kommutativ!Gruppe}%
\index{Gruppe!kommutativ}%
oder \emph{abelsch},
\index{abelsch}%
wenn die Gruppenoperation kommutativ ist.
In einer abelschen Gruppe wird die Gruppenoperation oft additiv geschrieben
und das neutrale Element mit $0$ bezeichnet.
\end{definition}

\begin{beispiel}
Auf der Menge $\mathbb{N}$ ist die Addition definiert und $0\in\mathbb{N}$ 
ist das neutrale Element bezüglich der Addition.
Trotzdem ist $\mathbb{N}$ keine Gruppe, da kein von 0 verschiedenes Element
ein Inverses hat.
In der Menge $\mathbb{Z}$ gibt es zu jeder Zahl $n\in\mathbb{N}$ das
inverse Element $-n\in\mathbb{Z}$, daher ist $\mathbb{Z}$ eine
abelsche Gruppe bezüglich der Addition.
\end{beispiel}

\begin{beispiel}
Die reellen Zahlen $\mathbb{R}$ bilden bezüglich der Addition eine
abelsche Gruppe mit dem neutralen Element $0\in\mathbb{R}$.
Die positiven reellen Zahlen $\mathbb{R}^+$ bilden bezüglich der
Multiplikation eine abelsche Gruppe mit neutralem Element
$1\in\mathbb{R}^+$.
Die Exponentialfunktion
\[
\exp
\colon
\mathbb{R}\to\mathbb{R}^+
:
x\mapsto e^x
\]
bildet die Addition in $\mathbb{R}$ in die Multiplikation in $\mathbb{R}^+$
ab, es gilt
\begin{align*}
\exp(a+b) &= \exp(a)\exp(b),
&
\exp(0)&=1
&&\text{und}
&
\exp(a)^{-1} &= \exp(-a).
\end{align*}
Eine solche Abbildung heisst ein \emph{Gruppenhomomorphismus}.
\index{Homomorphismus!Gruppen-}%
\index{Gruppenhomomorphismus}%
\end{beispiel}

\begin{beispiel}
Die Menge 
\[
\operatorname{GL}_n(\mathbb{R})
=
\{
A\in M_{n\times n}(\mathbb{R})
\mid
\text{$A$ ist invertierbar}
\}
\]
ist eine Gruppe bezüglich der Matrizenmultiplikation mit der
Einheitsmatrix $I$ als neutralem Element.
Da die Matrizenmultiplikation für $n>1$ nicht kommutativ ist,
ist $\operatorname{GL}_n(\mathbb{R})$ eine Gruppe.
\end{beispiel}

\begin{beispiel}
Die Menge der Matrizen
\[
\operatorname{SL}_n(\mathbb{R})
=
\{
A\in M_{n\times n}(\mathbb{R})
\mid
\det A=1
\}
\]
mit Determinante $1$ ist eine Gruppe, da alle Matrizen mit Determinante
$\ne 0$ invertierbar sind.
Es ist also 
$
\operatorname{SL}_n(\mathbb{R})
\subset
\operatorname{GL}_n(\mathbb{R})
$.

Sogar die Menge
\[
\operatorname{SL}_n(\mathbb{Z})
=
\{
A\in M_{n\times n}(\mathbb{Z})
\mid
\det A = 1
\}
\]
ist eine Gruppe, denn die Einträge der inversen Matrix haben den gemeinsamen
Nenner $\det A$, während der Zähler die Determinante einer Minormatrix ist.
Wegen $\det A=1$ folgt, dass die Einträge von $A^{-1}$ ebenfalls ganzahlig
sind.
\end{beispiel}

Die vertrauten Zahlenmengen zeichnen sich dadurch aus, dass sie nicht
nur eine Operation tragen, sondern zusätzlich zur Addition oft auch
eine Multiplikation.

\begin{definition}[Ring]
Ein \emph{Ring} ist eine Menge $R$ mit zwei Operationen,
der Addition $+$ und der Multiplikation $\cdot$, mit den folgenden
Eigeschaften:
\begin{enumerate}
\item
Es gibt ein neutrales Element $0\in R$ der Addition so, dass $R$
bezüglich der Addition eine abelsche Gruppe ist.
\item
Die Multiplikation ist assoziativ und \emph{distributiv}:
\index{distributiv}%
\begin{align*}
a(bc) &= (ab)c
\\
a(b+c)&= ab + ac & (a+b)c &= ac + bc.
\end{align*}
\item
$R$ heisst \emph{Ring mit Eins}, wenn es in $R^*=R\setminus\{0\}$ ein
ein neutrales Element $1\in R^*$ der Multiplikation gibt.
\item
Der Ring $R$ heisst \emph{kommutativ}, wenn die Multiplikation kommutativ ist.
\index{kommutativ}%
\item
Eine \emph{Einheit} in $R$ ist eine Element $u\in R^*$, welches bezüglich
\index{Einheit}%
der Multiplikation ist.
\item
Ein Element $a\in R^*$ heisst ein \emph{Nullteiler}, wenn es ein Element
\index{Nullteiler}%
$b\in R^*$ gibt, sodass $ab=0$ ist.
\end{enumerate}
\end{definition}

Nullteiler sind Elemente in $R^*$, die bezüglich der Multiplikation
nicht invertierbar sind.
Ist nämlich $ab=0$ und wäre $a^{-1}$ ein zu $a$ inverses Element, dann
wäre
\[
a^{-1}0
=
a^{-1}ab
=
b,
\]
ein Widerspruch zu $b\in R^*$.

\begin{beispiel}
Die Menge $\mathbb{Z}$ der ganzen Zahlen ist ein Ring mit Eins.
Die einzigen Elemente, die bezüglich der Multiplikation invertierbar
sind, sind $\pm 1$.
In $\mathbb{Z}$ gibt es keine Nullteiler.
\end{beispiel}

\begin{beispiel}
Sei $n\in\mathbb{N}$ eine natürliche Zahl und
\[
\mathbb{Z}_n
=
\{
0,\dots, n-1
\}
\]
die Menge der Reste bei Teilung durch $n$.
Die Addition und die Multiplikation von ganzen Zahlen überträgt sich
entsprechende Operationen auf $\mathbb{Z}_n$, sodass $\mathbb{Z}_n$
ebenfalls ein kommutativer Ring mit $1$ ist.
\end{beispiel}

\begin{beispiel}
Die Menge $M_{n\times n}(\mathbb{R})$ der $n\times n$-Matrizen mit
reellen Einträgen ist ein Ring mit Eins bezüglich der Addition und
Multiplikation von Matrizen.
Für $n>1$ ist die Matrizenmultiplikation nicht kommutativ, daher
ist $M_{n\times n}(\mathbb{R})$ ein nicht-kommutativer Ring.

Für $n>1$ hat $M_{n\times n}(\mathbb{R})$ viele Nullteiler,
Zum Beispiel gilt
\[
\begin{pmatrix}
1&0\\0&0
\end{pmatrix}
\begin{pmatrix}
0&0\\0&1
\end{pmatrix}
=
\begin{pmatrix}
0&0\\
0&0
\end{pmatrix}
\]
die beiden Matrizen links sind also Nullteiler und können nicht
invertiert werden.
\end{beispiel}

Die Mengen $\mathbb{R}$ und $\mathbb{F}_p$ zeichen sich dadurch
aus, dass nicht nur jedes Element ein additives Inverses hat, sondern
auch jedes von $0$ verschiedene Element ein multiplikatives Inverses hat.
Eine solche Menge heisst ein Körper.

\begin{definition}[Körper]
Ein Körper $K$ ist ein Ring mit Eins derart, dass $K^*$ eine abelsche
Gruppe bezüglich der Multiplikation ist.
\end{definition}

\begin{beispiel}
Die Menge $\mathbb{Q}$ der rationalen Zahlen ist ein Körper.
\end{beispiel}

\begin{beispiel}
Die Menge $\mathbb{F}_p$ ist für jede Primzahl ein Körper.
\end{beispiel}

%
% Körperaxiome für R
%
\subsubsection{Körperaxiome für $\mathbb{R}$}
Die reellen Zahlen werden als Grenzwerte von Cauchy-Folgen von
rationalen Zahlen konstruiert.
Sind $(a_n)_{n\in\mathbb{N}}$ und $(b_n)_{n\in\mathbb{N}}$ zwei
Cauchy-Folgen von rationalen Zahlen mit Grenzwerten $a$ bzw.~$b\in\mathbb{R}$.
In der Analysis lernt man, dass
\begin{align*}
\lim_{n\to\infty} (a_n+b_n)
&=
\lim_{n\to\infty} a_n + \lim_{n\to\infty} b_n
=
a+b
\\
\lim_{n\to\infty} (a_nb_n)
&=
\Bigl(\lim_{n\to\infty} a_n\Bigr) \cdot \Bigl(\lim_{n\to\infty} b_n\Bigr)
=
ab
\intertext{und falls $b_n\ne 0$ und $b\ne 0$ gilt auch}
\lim_{n\to\infty} \frac{1}{b_n}
&=
\frac{1}{\displaystyle\lim_{n\to\infty}b_n}
=
\frac{1}{b}
\\
\lim_{n\to\infty}\frac{a_n}{b_n}
&=
\frac{\displaystyle\lim_{n\to\infty} a_n}{\displaystyle\lim_{n\to\infty}b_n}
=
\frac{a}{b}.
\end{align*}
Diese Gleichungen zeigen, dass die Grenzwertbildung, durch die $\mathbb{R}$
definiert wird, mit den arithmetischen Operationen kompatibel sind.
In der Analysis spricht man davon, dass die arithmetischen Operationen
stetig sind.
Insbesondere folgt, dass $\mathbb{R}$ bezüglich der Addition eine
abelsche Gruppe ist und $\mathbb{R}^*$ bezüglich der Multiplikation
eine multiplikative Gruppe.
$\mathbb{R}$ ist also ein Körper, der die rationalen Zahlen als
echten Teilkörper enthält.

%
% Körperaxiome für eine algebraische Erweiterung
%
\subsubsection{Körperaxiome für eine algebraische Erweiterung}
Sei $K$ ein Körper und $m\in K[x]$ ein irreduzibles 
Polynom, welches 
\[
m(x)
=
x^n + m_{n-1}x^{n-1} + \dots + m_1x + m_0
\]
geschrieben werden soll.
Der Körper $K$ kann durch ein Element $\alpha$, welches die
Identität
\begin{equation}
m(\alpha)
=
\alpha^n
+
m_{n-1}\alpha^{n_1}
+
\dots
+
m_1\alpha
+
m_0
\label{buch:komplex:zahlen:eqn:alphaidentitaet}
\end{equation}
erfüllt, erweitert werden.
Dazu werden die Rechenregeln von Polynomen in $\alpha$ mit Koeffizienten
aus $K$ mit der Identität~\eqref{buch:komplex:zahlen:eqn:alphaidentitaet}
reduziert.
Wie früher gezeigt wurde, entsteht so die Menge
\[
K(\alpha)
=
\{
a_0+a_1\alpha + \dots a_{n-1}\alpha^{n-1}
\mid
a_0,\dots,a_{n-1}\in K
\},
\]
die ein $n$-dimensionalers Vektorraum über dem Körper $K$.
Die bereits früher konstruierte Definition der Multiplikation
macht aus $K(\alpha)$ eine Körper, in dem $\alpha$ eine Nullstelle
des Polynoms $m$ ist.

%
% Der Körper C = R(i)
%
\subsubsection{Die Körper $\mathbb{C}=\mathbb{R}(i)$ und $\mathbb{Q}(i)$}
Sei $m = x^2 + 1 \in\mathbb{R}[x]$.
Dieses Polynom ist irreduzibel und definiert daher eine algebraische
Erweiterung von $\mathbb{R}$, die mit $\mathbb{R}(i)$ bezeichnet wird.
Das Element $i$ ist eine Nullstelle von $m$, es gilt also $i^2+1=0$.
Als algebraische Erweiterung von $\mathbb{R}$ ist $\mathbb{R}(i)$ ein
Körper, er heisst der \emph{Körper der komplexen Zahlen}.
\index{komplexe Zahlen}%
Das Element $i$ heisst die \emph{imaginäre Einheit}.
\index{imaginäre Einheit}%
\index{i@$i$}%

Die Menge $\mathbb{R}(i)$ besteht aus den Zahlen
\[
\mathbb{R}(i)
=
\{
a+bi
\mid
a,b\in\mathbb{R}
\}
\]
mit den Ringoperationen
\begin{align*}
\text{Addition:}&&
(a+bi) \pm (c+di)
&=
(a\pm c) + (b\pm d)i
\\
\text{Multiplikation:}&&
(a+bi)(c+di)
&=
ac-bd + (ad+bd)i.
\end{align*}
Für das zu $c+di$ inverse Element muss man wie früher gezeigt
mit $c-di$ erweitern und erhält
\begin{align*}
\frac{1}{c+di}
&=
\frac{1}{c+di}\cdot\frac{c-di}{c-di}
=
\frac{c-di}{c^2+d^2}
=
\frac{c}{c^2+d^2}
+
\frac{-d}{c^2+d^2}i
\\
\text{und}
\qquad
\frac{a+bi}{c+di}
&=
\frac{ac+bd + (-ad+bc)i}{c^2+d^2}
+
\frac{ac+bd}{c^2+d^2}
+
\frac{-ad+bc}{c^2+d^2}i
.
\end{align*}
Die Operationen zur Berechnung des inversen Elementes sind 
für sich genommen nützlich, wir geben ihnen in der folgenden
Definition einen Namen.

\begin{definition}
Für $z=a+bi\in\mathbb{C}$ heisst
$\bar z = a-bi$ die zu $z$ \emph{konjugiert komplexe} Zahl.
\index{konjugiert komplex}%
\end{definition}

Das Polynom $m=x^2+1\in\mathbb{Q}[x]$ ist natürlich auch als
Polynom mit rationalen Koeffizienten irreduzibel.
Daher lässt sich auch $\mathbb{Q}$ durch die imaginäre Einheit
$i$ algebraisch zum Körper $\mathbb{Q}(i)$ erweitern.
Im Gegensatz zu $\mathbb{C}$ lässt sich in $\mathbb{Q}(i)$
exakt rechnen, es ist daher für Computeralgebrasysteme 
unerlässlich.

Die Gleichung $x^2+2=0$ hat in $\mathbb{C}$ die Lösungen
\[
x=
\pm
i
\!\sqrt{2}
\in\mathbb{C},
\]
die allerdings nicht in $\mathbb{Q}(i)$ ist.
Daher muss der Körper $\mathbb{Q}(i)$ erst um $\!\sqrt{2}$ 
algebraisch zu $\mathbb{Q}(i,\!\sqrt{2})$ erweitert werden.
In diesem Körper findet man die beiden Lösungen der Gleichung $x^2-2=0$.

Man beachte, dass sich die Zahlen $i$ und $-i$ mit algebraischen
Mitteln nicht unterscheiden lassen.
Für Zahlen mit reeller Quadratwurzel kann man dank der Ordnungsrelation
in $\mathbb{R}$ die postive von der negativen Quadratwurzel unterschieden.
Für komplexe Zahlen gibt es keine Vergleichsrelation

%
% Matrixschreibweise für komplexe Zahlen
%
\subsection{Matrixschreibweise für komplexe Zahlen}

\subsubsection{Algebraische Erweiterungen als Matrizen}

\subsubsection{Die imaginäre Einheit als Matrix}

\subsubsection{Quaternionen}

%
% Geometrische Beschreibung der Rechenoperationen in C
%
\subsection{Geometrische Beschreibung der Rechenoperationen in $\mathbb{C}$}

