%
% 2-funktionen.tex
%
% (c) 2025 Prof Dr Andreas Müller
%
\section{Komplexe Funktionen
\label{buch:komplex:section:funktionen}}
\kopfrechts{Komplexe Funktionen}
Die Algebra der komplexen Zahlen ermöglicht Potenzen und Polynome
zu konstruieren.
Wir können dieses als Funktionen mit komplexwertigen Argumenten
und komplexen Werten betrachten.
In diesem Abschnitt konstruieren wir weitere Funktionen und entwickeln
Techniken, um komplexe Funktionen zu visualisieren.

%
% Komplexe Funktionen als Abbildungen
%
\subsection{Komplexe Funktionen als Abbildungen}
Von einer reelle Funktion
$f\colon \mathbb{R}\to\mathbb{R}$
oder
$f\colon I\to\mathbb{R}$ mit Definitionsbereich $I\subset\mathbb{R}$
kann man sich leicht mit Hilfe eines Graphen ein Bild machen.
Er stellt die Abhängigkeit zwischen Argument $x$ aus dem
Definitionsbereich und dem Wert $f(x)$ als Kurve bestehend aus
den Punkten $(x,f(x))$ dar.
Für komplexe Funktionen ist sowohl das Argument $z$ wie auch der
Wert $f(z)$ eine komplexe Zahl, die beide einen zweidimensionalen
Raum zur Visualisierung benötigen.
Im dreidimensionalen Raum ist also schlicht nicht genug Platz vorhanden,
um komplexe Funktionen als Grahen zu visualisieren.

%
% Definitions- und Wertebereich
%
\subsubsection{Definitions- und Wertebereich}
Nur in Ausnahmefällen wird eine komplexe Funktion für beliebige Argument
$z\in\mathbb{C}$ definiert sein.
Meistens wird die Funktion nur auf einer Teilmenge
$U\subset \mathbb{C}$ definiert sein.
Um später die Ableitung einer komplexen Funktion definieren zu können,
müssen zusätzliche Annahmen über den Definitionsbereich gemacht werden.
Die Ableitung approximiert die Änderung des Funktionswertes $f(z)$
in Abhängigkeit von kleinen Abweichungen $\Delta z$ vom Ausgangspunkt $z$
als lineare Funktion von $\Delta z$.
Diese kann nur ermittelt werden, wenn es möglich ist, Änderungen
$\Delta z$ in jede beliebige Richtung innerhalb der komplexen Ebene
zu untersuchen.
Dazu muss der Definitionsbereich eine kleine Umgebung des Punktes $z$
enthalten.

\begin{definition}[offene Menge, Gebiet]
Eine Teilmenge $U\subset \mathbb{C}$ heisst \emph{offen}, wenn es zu
\index{offen}%
jedem Punkt $z\in U$ ein $\varepsilon > 0$ gibt derart, dass die
$\varepsilon$-Umgebung
\[
U_{\varepsilon}(z)
=
\{
z'\in\mathbb{C}
\mid
|z-z'|<\varepsilon
\}
\]
von $z$ ebenfalls in $U$ enthalten ist, also $U_\varepsilon(z)\subset U$.
Eine offene Teilmenge von $\mathbb{C}$ heisst auch ein \emph{Gebiet}.
\index{Gebiet}%
\end{definition}

\begin{beispiel}
Die Funktion der \emph{komplexen Konjugation}
\index{Konjugation}%
\[
\bar{\phantom{o}}
\colon
\mathbb{C} \to \mathbb{C}
:
z\mapsto \bar{z}
\]
ist eine Abbildung, die auf allen komplexen Zahlen definiert ist.
Schreibt man $z=x+iy$, dann ist $\overline{x+iy}=x-iy$.
Geometrisch ist die komplexe Konjugation die Spiegelung an der
reellen Achse.
\end{beispiel}

\begin{beispiel}
Die Abbildung $z\mapsto z$ ist auf ganz $\mathbb{C}$ definiert,
sie wird auch die \emph{identische Abbildung} von $\mathbb{C}$
genannt und $\operatorname{id}_{\mathbb{C}}$ geschrieben:
$\operatorname{id}_{\mathbb{C}}(z)=z$.
\end{beispiel}

\begin{beispiel}
Die Funktion
\[
f
\colon
\mathbb{C}\setminus\{1\}
\to
\mathbb{C}
:
z\mapsto + \frac{1}{z-1}
\]
ist überall ausser im Punkt $1\in\mathbb{C}$ definiert.
Die Menge $U=\mathbb{C}\setminus\{1\}$ ist eine offene Menge, denn
für jeden Punkt $z\ne 1$ können wir $\varepsilon = |z-1|$ setzen
und uns davon überzeugen, dass $U_{\varepsilon}(z)\subset U$ ist.
Für einen Punkt $z'\in U_{\varepsilon}(z)$ gilt
\[
|z-z'|<\varepsilon = |z-1|
\qquad\Rightarrow\qquad
z'\ne 1,
\]
also kann $U_\varepsilon(z)$ den Punkt $1$ nicht enthalten.
Da $1$ der einzige Punkt ist, der nicht in $U$ ist, folgt
$U_\varepsilon(z)\subset U$.
\end{beispiel}

%
% Real- und Imaginaerteil
%
\subsubsection{Real- und Imaginärteil}
In diesem Abschnitt sei $f\colon U\to\mathbb{C}$ eine komplexe Funktion
auf einer offenen Teilmenge $U\subset\mathbb{C}$.
Das Argument $z\in U$ können wir auch als Real- und Imaginärteil
schreiben.
Wir verwenden die Notation $x+iy$, sodass der Realteil die
$x$-Koordinate in der gaussschen Zahlenebene ist und der Imaginärteil
die $y$-Koordinate.
Die komplexe Funktion $f(z)=f(x+iy)$ wird somit zu einer Funktion
von zwei reellen Variablen, die auf der Menge
\[
V
=
\{
(x,y)
\mid
x+iy\in U
\}
\]
definiert ist.
Hält man die $y$-Koordinaten fest, definiert $x\mapsto f(x+iy)$ 
eine durch $x$ parametriesierte Kurve in $\mathbb{C}$,
ebenso ist $y\mapsto f(x+iy)$ für festes $x$ eine Kurve in $\mathbb{C}$.
Die Abbildung $f$ bildet also Gitterlinien im Definitionsbereich auf
Kurven in $\mathbb{C}$ ab.
%
% fig-kehrwert.tex
%
% (c) 2026 Prof Dr Andreas Müller
%
\begin{figure}
\centering
\includegraphics{chapters/010-komplex/images/kehrwert.pdf}
\caption{Visualisierung der komplexen Funktion $f(z) = 1/(1+z)$ 
als Abbildung der Koordinatenlinien im Definitionsbereich.
Die Koordinatenlinien zu konstantem $y$ werden durch $x\mapsto f(x+iy)$ 
auf die roten Kurven abgebildet und die Koordinatenlinien $x$ durch
$y\mapsto f(x+iy)$ auf die blauen.
\label{buch:komplex:funktionen:fig:kehrwert}}
\end{figure}
%
Die Funktion $f$ kann also visualisiert werden durch das Bild der
Gitterlinien in $\mathbb{C}$, wie dies in
Abbildung~\ref{buch:komplex:funktionen:fig:kehrwert}
für die Funktion $f(z) = 1/(z-z_0)$ gezeigt wird.
Es fällt auf, dass die Bildkurven der Koordinatenlinien alle Kreise
zu sein scheinen.
Dieser Beobachtung wird Abschnitt~\ref{buch:komplex:section:moebius}
auf den Grund gehen.

Die Werte der Funktionen $f(z)$ sind komplexe Zahlen, die wir 
in Real- und Imaginärteil aufteilen können.
Wir schreiben
\[
f(z)
=
f(x+iy)
=
u(x+iy) + iv(x+iy)
=
u(x,y) + iv(x,y).
\]
Im Falle der in 
Abbildung~\ref{buch:komplex:funktionen:fig:kehrwert}
dargestellten Funktion $f(z)=1/(z-z_0)$ sind die
beiden Funktionen $u(x,y)$ und $v(x,y)$  gegeben durch
\begin{align*}
u(x,y)
&=
\frac{x-x_0}{(x-x_0)^2 + (y-y_0)^2}
&&\text{und}&
v(x,y)
&=
-
\frac{y-y_0}{(x-x_0)^2 + (y-y_0)^2},
\end{align*}
wobei $x_0=\operatorname{Re}z_0$ und $y_0=\operatorname{Im}z_0$.

Die Funktionen $u(x,y)$ und $v(x,y)$ definieren zwei Scharen von
Kurven in der $z=x+iy$-Ebene.
Die Mengen
\[
U(u_0)
=
\{
z=x+iy
\mid
u(x,y) = u_0
\}
\qquad\text{und}\qquad
V(v_0)
=
\{
z=x+iy
\mid
v(x,y)
=
v_0
\}
\]
sind die Kurven im Definitionsbereich, die auf die Koordinatenlinien
$\{u_0+iy\mid y\in\mathbb{R}\}$ 
bzw.~$\{x+iv_0\mid x\in\mathbb{R}\}$
abgebildet werden.
Die Schnittpunkte in $U(u_0)\cap V(v_0)$ sind Lösungen der Gleichung
$f(z) = u_0+iv_0$.

\begin{beispiel}
%
% fig-hyperbeln.tex
%
% (c) 2026 Prof Dr Andreas Müller
%
\begin{figure}
\centering
\includegraphics{chapters/010-komplex/images/hyperbeln.pdf}
\caption{Urbildkurven der Gitterlinien unter der Abbildung
$z\mapsto f(z)=z^2$.
\label{buch:komplex:funktionen:fig:hyperbeln}}
\end{figure}
%
Wir betrachten die Funktion $z\mapsto z^2$, die durch die Funktionen
\[
u(x,y) = x^2-y^2
\qquad\text{und}\qquad
v(x,y) = 2xy
\]
beschrieben wird.
Die Kurven $u(x,y)=u_0$ sind Hyperbeln mit den Geraden $y=\pm x$, sie
können mit den Funktionen
\[
(x,y)
=
\begin{cases}
\phantom{-}\!\sqrt{u_0}
(\pm\cosh t, \sinh t)
&\qquad\text{für $u_0>0$}
\\
\!\sqrt{-u_0}
(\sinh t, \pm\cosh t)
&\qquad\text{für $u_0<0$}
\end{cases}
\]
beschrieben werden, die in
Abbildung~\ref{buch:komplex:funktionen:fig:hyperbeln}
dargestellt sind.
Die in 
Abbildung~\ref{buch:komplex:funktionen:fig:hyperbeln}
roten Hyperbeln $v(x,y)=v_0$ entstehen aus den blauen
Hyperbeln durch Drehung um $45^\circ$.
\end{beispiel}

%
% Betrag und Argument
%
\subsubsection{Betrag und Argument}
Die komplexen Zahlen können auch in der Polardarstellung
$z=r(\cos \varphi + i\sin\varphi)$ mit Betrag $r=|z|$ und
Argument $\varphi=\operatorname{arg}z$ parametrisiert werden.
Die zugehörigen Koordinatenlinien sind Kreise um den Nullpunkt
für Kurven konstanten Betrags und Geraden durch den Nullpunkt
für Kurven konstanten Argumentes.
Der Wert einer komplexen Funktion kann dann ebenfalls in Polardarstellung
als
\begin{align*}
f(z)
=
R(z) \bigl(\cos\Phi(z) + i\sin\Phi(z)\bigr)
&=
R(r,\varphi) \bigl(\cos\Phi(r,\varphi) + i\sin\Phi(r,\varphi)\bigr)
\\
&=
R(x,y)\bigl(\cos\Phi(x,y) + i\sin\Phi(x,y)\bigr)
\end{align*}
geschrieben werden.
Auch für die Polardarstellung gilt, dass Grenzwerte komponentenweise
oder durch Grenzübergang nur einer der beiden Koordinaten
berechnet werden können.

Die Polardarstellung eröffnet eine weiter Möglichkeit der
Visualisierung einer komplexen Funktion.
%
% fig-farben.tex
%
% (c) 2026 Prof Dr Andreas Müller
%
\begin{figure}
\centering
\includegraphics{chapters/010-komplex/images/farben.pdf}
\caption{Visualisierung einer komplexen Funktion in Polardarstellung
$f(z)=R(z) \bigl(\cos\Phi(z)+i\sin\Phi(z)\bigr)$ als Fläche der Punkte
$(x,y,R(x+iy))$ mit der Farbe, die durch den Winkel $\Phi(x+iy)$ auf
dem Farbkreis dargestellt wird.
\label{buch:komplex:funktionen:fig:farben}}
\end{figure}
%
Die Funktion $R(x,y)$ kann natürlich als Graph dargestellt
werden, eine Fläche, die aus den Punkten $(x,y,R(x,y))$
besteht.
Um auch die im Argument $\Phi(x,y)$ enthaltene Information
sichtbar zu machen, kann man den Winkel $\Phi(x,y)$ als Koordinate
auf einem Farbkreis betrachten und den Punkt $(x,y,R(x,y))$
der Fläche mit der gefundenen Farbe einfärben.
Es ist zwar nicht einfach, den exakten Wert des Arguments aus
der Farbe zu rekonstruieren, aber die Darstellung ist ausreichend,
um schnelle Änderungen des Arguments an schnellen Farbwechseln
zu erkennen.
Abbildung~\ref{buch:komplex:funktionen:fig:farben}
visualisiert die Funktion 
\[
f(z)
=
\frac{1}{z-1}
+
\frac{1}{z+1}
\]
auf diese Weise.

%
% Grenzwerte und Stetigkeit
%
\subsubsection{Grenzwerte und Stetigkeit}
Da die Betragsfunktion sowohl im Definitions- wie im Wertebereich
einer stetigen Funktion einen Begriff des Abstandes von Punkten
definiert, können jetzt wie für komplexe Funktionen den Begriff des
Grenzwertes definieren.
Wir beginnen damit, Grenzwerte für komplexe Zahlenfolgen zu definieren.

\begin{definition}[Grenzwert]
Sei $(z_n)_{n\in\mathbb{N}}$ eine Folge komplexer Zahlen.
Sie heisst eine \emph{Cauchy-Folge}, wenn für jedes $\varepsilon>0$ ein
$N\in\mathbb{N}$ existiert derart, dass $|z_n-z_m|< \varepsilon$ 
für $n,m>N$.
Die komplexe Zahl $z$ heisst \emph{Grenzwert} der Folge, wenn es für
jedes $\varepsilon>0$ ein $N\in\mathbb{N}$ derart, dass $|z-z_n|<\varepsilon$
ist $n>N$.
\end{definition}

Der Grenzwert einer komplexen Zahlenfolge kann für Real- und Imaginärteil
separat durchgeführt werden.
Wegen
\[
|\operatorname{Re}z_n - \operatorname{Re}z| \le |z_n-z_m|
\qquad\text{und}\qquad
|\operatorname{Im}z_n - \operatorname{Im}z| \le |z_n-z_m|
\]
sind auch die Folgen $(\operatorname{Re}z_n)_{n\in\mathbb{N}}$
und $(\operatorname{Im}z_n)_{n\in\mathbb{N}}$ Cauchy-Folgen
von reellen Zahlen.
Da $\mathbb{R}$ als Verfollständigung definiert worden war, haben
die beiden Folgen Grenzwerte in $\mathbb{R}$, die wir mit
\begin{equation}
\lim_{n\to\infty}\operatorname{Re}z_n = x
\qquad\text{und}\qquad
\lim_{n\to\infty}\operatorname{Im}z_n = y.
\label{buch:komplex:funktionen:eqn:reellgrenzwert}
\end{equation}
Wir vermuten, dass $z=x+iy$ der Grenzwert der Folge $z_n$ ist.
Sei $\varepsilon>0$ gegeben.
Dann gibt es wegen
\eqref{buch:komplex:funktionen:eqn:reellgrenzwert}
eine Zahl $N\in\mathbb{N}$ derart, dass
\[
|\operatorname{Re}z_n-x|<\frac{\varepsilon}{2}
\qquad\text{und}\qquad
|\operatorname{Im}z_n-y|<\frac{\varepsilon}{2}
\]
für $n>N$ gilt.
Dann ist
\begin{align*}
|z-z_n|
&=
|x+iy-\operatorname{Re}z_n -i\operatorname{Im}z_n|
\\
&=
|
(x -\operatorname{Re}z_n)
+
(iy -i\operatorname{Im}z_n)
|
\\
&<
|x -\operatorname{Re}z_n|
+
|iy -i\operatorname{Im}z_n|
<
\frac{\varepsilon}2
+
\frac{\varepsilon}2
<
\varepsilon,
\end{align*}
die Folge $(z_n)_{n\in\mathbb{N}}$ konvergiert also gegen $z$.
Wir fassen dies im folgenden Satz zusammen.

\begin{satz}[Grenzwert einer Cauchy-Folge in $\mathbb{C}$]
Ist $(z_n)_{n\mathbb{N}}$ eine Cauchy-Folge in $\mathbb{C}$, dann konvergiert
sie gegen
\[
z
=
\lim_{n\to \infty} \operatorname{Re}z_n
+
i
\lim_{n\to \infty} \operatorname{Im}z_n
=
\lim_{n\to\infty}z_n.
\]
\end{satz}

Mit dem Abstand in $U\subset\mathbb{C}$  lässt sich jetzt auch der
Grenzwert einer Funktion in einem Punkt $z_0\in\mathbb{C}$ definieren,
der nicht notwendigerweise in $U$ liegen muss.

\begin{definition}[Grenzwert einer Funktion]
Sei $f\colon U\to\mathbb{C}$ eine komplexe Funktion definiert auf der
Menge $U\subset\mathbb{C}$ und sei $z_0\in \overline{U}$.
Die Zahl $a\in\mathbb{C}$ heisst Grenzwert von $f$ für $z\to z_0$,
wenn für jedes $\varepsilon>0$ ein $\delta>0$ existert derart,
dass für alle $z\in U$ mit $|z-z_0|<\delta$
folgt, dass $|f(z)-a|<\varepsilon$.
\end{definition}

Man beiachte, dass es keine Rolle spielt, aus welcher Richtung das
Argment $z$ der komplexen Funktion $f$ sich dem Punkt $z_0$ nähert.
Insbesondere lässt sich der Grenzwert einer komplexen Funktion,
falls er existiert, dadurch berechnen, dass sich dem Punkt $z_0$
nur in reeller oder nur in imaginärer Richtung nähert.
Wenn der Grenzwert existiert, dann gilt
\[
\lim_{z\to z_0} f(z) = a
\quad\Rightarrow\quad
\lim_{z\to z_0} f(z)
=
\begin{cases}
\;\displaystyle
\lim_{x\to 0} f(z_0+x) = a \\
\;\displaystyle
\lim_{y\to 0} f(z_0+iy) = a.
\end{cases}
\]

\begin{definition}[Stetigkeit]
Eine komplexe Funktion $f\colon U\to\mathbb{C}$ heisst
\emph{stetig im Punkt} $z_0\in U$, wenn für jedes $\varepsilon >0$
ein $\delta>0$ existiert derart, dass $|f(z)-f(z_0)|<\varepsilon$
ist, sofern $|z-z_0|<\delta$.
Die Funktion $f$ heisst \emph{stetig}, wenn sie in jedem Punkt von $U$
stetig ist.
\end{definition}

Offenbar ist Stetigkeit einer Funktion in einem Punkt $z_0$ gleichbedeutend
damit, dass der Grenzwert der Funktion in diesem Punkt mit dem Funktionswert
übereinstimmt, also
\[
\lim_{z\to z_0} f(z) = f(z_0).
\]

Für reelle Funktionen wird oft auch der Begriff der divergenten Funktionen
und Folgen definiert.
Zum Beispiel schreibt man für die Funktion $f(x) = 1/x^2$, dass
\[
\lim_{x\to 0}
\frac{1}{x^2}
=
+\infty.
\]
Man meint damit, dass für jedes $M>0$ eine $\delta>0$ existiert derart,
dass
\[
f(x) > M
\qquad\text{wenn}\qquad
|x|<\delta.
\]
Die Definition benötigt als Grundlage die Ordnungsrelation von $\mathbb{R}$.
Sie ermöglicht auch die Unterscheidung von $+\infty$ und $-\infty$.
Da $\mathbb{C}$ keine vergleichbare Ordnungsrelation hat, wird der Begriff
der Konvergenz gegen einen Punkt im Unendlichen sinnlos.

Für die komplexwertige Funktion $z\mapsto 1/z^2$ zeigt sich die Unsinnigkeit
besonders klar, denn die Grenzwerte
\begin{align*}
\lim_{y\to 0} \frac{1}{(iy)^2}
&=
\lim_{y\to 0} \frac{1}{-y^2}
=
-\infty
\\
\lim_{x\to 0}\frac{1}{x^2}
&=
+\infty
\end{align*}
unterscheiden sich.
Die uneigentlichen Grenzwerte hängen also offenbar von der Richtung ab,
in der man sich der Polstelle bei $z=0$ nähert.

%
% Polynome und algebraische Funktionen
%
\subsection{Polynome und algebraische Funktionen}

%
% Potenzen
%
\subsubsection{Potenzen}

%
% Polynome
%
\subsubsection{Polynome}

%
% Algebraische Funktionen
%
\subsubsection{Algebraische Funktionen}

%
% Potenzreihen
%
\subsection{Potenzreihen}


