%
% 3-residuen.tex -- 3. Residuen
%
% (c) 2025 Prof Dr Andreas Müller
%
\section{Residuen
\label{buch:integration:section:residuen}}
\kopfrechts{Residuen}
Der Residuensatz ermöglicht, reelle Integrale zu berechnen, denen 
allein mit den Mitteln der reellen Analysis nur schwer beizukommen
ist.
Die Vorgehensweise ist dabei immer die Gleiche.
Das Integrationsintervall ist eine Strecke in der komplexen Ebene.
Sie wird durch weitere Wegstücke zu einem geschlossenen Weg
vervollständigt.
Die zusätzlichen Wegstücke werden so gewählt, dass sie sich
einfach berechnen lassen.
Das Integral über den geschlossenen Weg kann auch mit dem Residuensatz
bestimmt werden, das gesuchte Integral ergibt sich dann aus der Differenz.





