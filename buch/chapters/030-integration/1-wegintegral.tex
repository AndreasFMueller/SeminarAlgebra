%
% 1-wegintegral.tex -- 1. Wegintegral
%
% (c) 2025 Prof Dr Andreas Müller
%
\section{Wegintegrale in $\mathbb{C}$
\label{buch:integration:section:wegintegral}}
\kopfrechts{Wegintegrale in $\mathbb{C}$}
Das Riemann-Integral einer Funktion ist definiert für Invervalle.
Mit Hilfe eines Weges im Definitionsgebiet einer komplexen Funktion
können wir die Funktion in eine komplexwertige Funktion auf einem
Intervall, nämlich dem Definitionsbereich des Weges verwandeln.
Da es zwischen zwei Punkten von $U$ beliebig viele verschiedene
Wege gibt, müssen die Bedingungen identifiziert werden, unter denen
das Integral nicht von der Wahl des Weges abhängt.

%
% Wege in $\mathbb{C}$
%
\subsection{Wege in $\mathbb{R}^n$ und $\mathbb{C}$}
Ein \emph{Weg} in einer offenen Menge $U\subset \mathbb{R}^n$ ist eine
stetige Abbildung
\[
\gamma : I \to U : t \mapsto \gamma(t)
\]
für ein Intervall $I=[a,b]$.
Ein Weg heisst \emph{differenzierbarer Weg}, wenn die Funktion $\gamma$
differenzierbar ist.
Die Ableitung
\[
\dot{\gamma}
\colon
I
\to
\mathbb{R}^n
:
t
\mapsto
\dot{\gamma}(t)
=
\frac{d\gamma}{dt}(t)
\]
heisst auch der Tangentialvektor des Weges im PUnkt $\gamma(t)$.

Da $\mathbb{C}$ als zweidimensionaler reeller Vektorraum betrachtet
werden kann, ist ein Weg in $U\subset\mathbb{C}$ eine Abbildung
von einem Intervall $I$ nach $U$:
\[
\gamma
\colon
I
\to
U
:
t
\mapsto
\gamma(t).
\]
Die Ableitung nach $t$ ist ebenfalls eine komplexe Zahl, die Funktion
\[
\dot{\gamma}
\colon
I
\to
\mathbb{C}
:
t
\mapsto
\dot{\gamma}(t)
=
\frac{d\gamma}{dt}(t)
\]
heisst wieder Tangentialvektor, er ist jetzt aber auch eine komplexe 
Zahl.

Ein Weg in $\mathbb{R}^2$ entsteht dadurch, dass man den Vektor aus
Real- und Imaginärteil von $\gamma(t)$ bildet:
\[
\gamma_{\mathbb{R}}
\colon
I
\to
\mathbb{R}^2
:
t
\mapsto
\begin{pmatrix}
\operatorname{Re}\gamma(t)\\
\operatorname{Im}\gamma(t)
\end{pmatrix}.
\]
Der Tangentialvektor besteht aus den Ableitungen von Real-
und Imaginärteil
\[
\dot{\gamma}
\colon
I
\to
\mathbb{C}
:
t
\mapsto
\frac{d}{dt}\bigl(\operatorname{Re}\gamma(t)\bigr)
+
i
\frac{d}{dt}\bigl(\operatorname{Im}\gamma(t)\bigr)
\]
nach $t$.

%
% fig-wege.tex
%
% (c) 2026 Prof Dr Andreas Müller
%
\begin{figure}
\centering
\includegraphics[width=\textwidth]{chapters/030-integration/images/wege.pdf}
\caption{Wege in der komplexen Ebene.
Links der Weg von Beispiel~\ref{buch:integration:wege:bsp:strecke},
der die Punkte $z_0$ und $z_1$ miteinander verbindet.
Rechts der geschlossene Weg von Beispiel~\ref{buch:integration:wege:bsp:kreis}.
\label{buch:integration:wege:fig:wege}}
\end{figure}
%

\begin{beispiel}
\label{buch:integration:wege:bsp:strecke}
Sei $U=\mathbb{C}$ und seien zwei Punkte $z_0,z_1\in \mathbb{C}$ gegeben.
Dann ist
\[
\gamma
\colon
[0,1]
\to
\mathbb{C}
:
(1-t)z_0 + tz_1
\]
ein differenzierbarer Weg, der $\gamma(0)=z_0$ mit $\gamma(1)=z_1$
verbindet.
Die Ableitung von $\gamma$ ist
\[
\dot{\gamma}(t)
=
z_1-z_0.
\]
Insbesondere ist die Ableitung konstant.

Weitere Wege mit den gleichen Endpunkten können konstruiert werden,
indem ein Weg addiert wird, der von $0$ nach $0$ führt.
Zum Beispiel ist für jedes $n\in\mathbb{N}$ 
\[
\delta(t)
=
r_0
(e^{2\pi int} - 1)
\]
ein kreisförmiger Weg mit Radius $r_0$, der $n$ Mal durchlaufen wird.
Fügt man diesen Weg und zustzlich den Summanden $i\sin \pi t$ zum Weg
$\gamma(t)$ hinzu, erhält man den Weg
\[
\gamma_1(t)
=
\gamma(t)
+
r_0
(e^{2\pi int} - 1)
+
i\sin \pi t,
\]
der in Abbildung~\ref{buch:integration:wege:fig:wege}~a)
{\color{darkgreen}grün} eingezeichnet ist.
Der zugehörige Geschwindigkeitsvektor 
\[
\dot{\gamma}_1(t)
=
(z_1-z_0)
+
2\pi i
n
r_0
e^{2\pi int}
+
i\pi\cos \pi t.
\qedhere
\]
\end{beispiel}

\begin{beispiel}
\label{buch:integration:wege:bsp:kreis}
Sei $U=\mathbb{C}$ und $r\in\mathbb{R}$.
Der Weg
\[
\gamma
\colon
[0,2\pi]
\to
U
:
t
\mapsto
r(\cos t + i\sin t)
\]
ist ein differenzierbarer Weg mit $\gamma(0) = \gamma(2\pi) = 1$.
Die Ableitung ist
\[
\dot{\gamma}(t)
=
r(-\sin t + i\cos t).
\]
In \eqref{XXX} 
wurde gezeigt, dass $\gamma(t) = re^{it}$ ist mit der Ableitung
$\dot{\gamma}(t)=rie^{it}$.
\end{beispiel}


%
% Integral entlang eines Weges
%
\subsection{Integral entlang eines Weges}

%
% Homotopie
%
\subsection{Homotopie}

