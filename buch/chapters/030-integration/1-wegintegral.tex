%
% 1-wegintegral.tex -- 1. Wegintegral
%
% (c) 2025 Prof Dr Andreas Müller
%
\section{Wegintegrale in $\mathbb{C}$
\label{buch:integration:section:wegintegral}}
\kopfrechts{Wegintegrale in $\mathbb{C}$}
Das Riemann-Integral einer Funktion ist definiert für Invervalle.
Mit Hilfe eines Weges im Definitionsgebiet einer komplexen Funktion
können wir die Funktion in eine komplexwertige Funktion auf einem
Intervall, nämlich dem Definitionsbereich des Weges verwandeln.
Da es zwischen zwei Punkten von $U$ beliebig viele verschiedene
Wege gibt, müssen die Bedingungen identifiziert werden, unter denen
das Integral nicht von der Wahl des Weges abhängt.

%
% Wege in $\mathbb{C}$
%
\subsection{Wege in $\mathbb{R}^n$ und $\mathbb{C}$}
Ein \emph{Weg} in einer offenen Menge $U\subset \mathbb{R}^n$ ist eine
stetige Abbildung
\[
\gamma : I \to U : t \mapsto \gamma(t)
\]
für ein Intervall $I=[a,b]$.
Ein Weg heisst \emph{differenzierbarer Weg}, wenn die Funktion $\gamma$
differenzierbar ist.
Die Ableitung
\[
\dot{\gamma}
\colon
I
\to
\mathbb{R}^n
:
t
\mapsto
\dot{\gamma}(t)
=
\frac{d\gamma}{dt}(t)
\]
heisst auch der Tangentialvektor des Weges im PUnkt $\gamma(t)$.

Da $\mathbb{C}$ als zweidimensionaler reeller Vektorraum betrachtet
werden kann, ist ein Weg in $U\subset\mathbb{C}$ eine Abbildung
von einem Intervall $I$ nach $U$:
\[
\gamma
\colon
I
\to
U
:
t
\mapsto
\gamma(t).
\]
Die Ableitung nach $t$ ist ebenfalls eine komplexe Zahl, die Funktion
\[
\dot{\gamma}
\colon
I
\to
\mathbb{C}
:
t
\mapsto
\dot{\gamma}(t)
=
\frac{d\gamma}{dt}(t)
\]
heisst wieder Tangentialvektor, er ist jetzt aber auch eine komplexe 
Zahl.

Ein Weg in $\mathbb{R}^2$ entsteht dadurch, dass man den Vektor aus
Real- und Imaginärteil von $\gamma(t)$ bildet:
\[
\gamma_{\mathbb{R}}
\colon
I
\to
\mathbb{R}^2
:
t
\mapsto
\begin{pmatrix}
\operatorname{Re}\gamma(t)\\
\operatorname{Im}\gamma(t)
\end{pmatrix}.
\]
Der Tangentialvektor besteht aus den Ableitungen von Real-
und Imaginärteil
\[
\dot{\gamma}
\colon
I
\to
\mathbb{C}
:
t
\mapsto
\frac{d}{dt}\bigl(\operatorname{Re}\gamma(t)\bigr)
+
i
\frac{d}{dt}\bigl(\operatorname{Im}\gamma(t)\bigr)
\]
nach $t$.

%
% fig-wege.tex
%
% (c) 2026 Prof Dr Andreas Müller
%
\begin{figure}
\centering
\includegraphics[width=\textwidth]{chapters/030-integration/images/wege.pdf}
\caption{Wege in der komplexen Ebene.
Links der Weg von Beispiel~\ref{buch:integration:wege:bsp:strecke},
der die Punkte $z_0$ und $z_1$ miteinander verbindet.
Rechts der geschlossene Weg von Beispiel~\ref{buch:integration:wege:bsp:kreis}.
\label{buch:integration:wege:fig:wege}}
\end{figure}
%

\begin{beispiel}
\label{buch:integration:wege:bsp:strecke}
Sei $U=\mathbb{C}$ und seien zwei Punkte $z_0,z_1\in \mathbb{C}$ gegeben.
Dann ist
\[
\gamma
\colon
[0,1]
\to
\mathbb{C}
:
(1-t)z_0 + tz_1
\]
ein differenzierbarer Weg, der $\gamma(0)=z_0$ mit $\gamma(1)=z_1$
verbindet.
Die Ableitung von $\gamma$ ist
\[
\dot{\gamma}(t)
=
z_1-z_0.
\]
Insbesondere ist die Ableitung konstant.

Weitere Wege mit den gleichen Endpunkten können konstruiert werden,
indem ein Weg addiert wird, der von $0$ nach $0$ führt.
Zum Beispiel ist für jedes $n\in\mathbb{N}$ 
\[
\delta(t)
=
r_0
(e^{2\pi int} - 1)
\]
ein kreisförmiger Weg mit Radius $r_0$, der $n$ Mal durchlaufen wird.
Fügt man diesen Weg und zustzlich den Summanden $i\sin \pi t$ zum Weg
$\gamma(t)$ hinzu, erhält man den Weg
\[
\gamma_1(t)
=
\gamma(t)
+
r_0
(e^{2\pi int} - 1)
+
i\sin \pi t,
\]
der in Abbildung~\ref{buch:integration:wege:fig:wege}~a)
{\color{darkgreen}grün} eingezeichnet ist.
Der zugehörige Geschwindigkeitsvektor 
\[
\dot{\gamma}_1(t)
=
(z_1-z_0)
+
2\pi i
n
r_0
e^{2\pi int}
+
i\pi\cos \pi t.
\qedhere
\]
\end{beispiel}

\begin{beispiel}
\label{buch:integration:wege:bsp:kreis}
Sei $U=\mathbb{C}$ und $r\in\mathbb{R}$.
Der Weg
\[
\gamma
\colon
[0,2\pi]
\to
U
:
t
\mapsto
r(\cos t + i\sin t)
\]
ist ein differenzierbarer Weg mit $\gamma(0) = \gamma(2\pi) = 1$.
Die Ableitung ist
\[
\dot{\gamma}(t)
=
r(-\sin t + i\cos t).
\]
In \eqref{XXX} 
wurde gezeigt, dass $\gamma(t) = re^{it}$ ist mit der Ableitung
$\dot{\gamma}(t)=rie^{it}$.
\end{beispiel}


%
% Integral entlang eines Weges
%
\subsection{Integral entlang eines Weges}
Für eine stetige komplexe Funktion $f(z)$ und einen Weg $\gamma(t)$
ist $f(\gamma(t))$ eine stetige Funktion der reellen Variablen $t$.
Nach allgemeinen Sätzen über das Riemann-Integral können wir also 
von Real- und Imaginärteil das Integral über das Definitionsgebiet
berechnen.
Die Definition eines Wegintegrals sollte aber möglichst nicht von
der Parametrisierung des Weges abhängen.
Das Integral der konstanten Funktion $1$ ist aber immer
\[
\int_a^b f(\gamma(t)) \,dt
=
\int_a^b 1\,dt
=
b-a,
\]
hängt also sehr wohl von der Parametrisierung ab.
Daher werden wir zunächst eine von der Parametrisierung unabhängige
Form des Wegintegrals definieren müssen.
Anschliessend werden wir zeigen, wie dieses Wegintegral mit der
Stammfunktion zusammenhängt.

%
% Definition des Wegintegrals
%
\subsubsection{Definition des Wegintegrals}
Das einführende Beispiel hat gezeigt, dass das Integral
von $f(\gamma(t))$ von der Parametrisierung abhängt.
Bewegt sich der Punkt $\gamma(t)$ langsamer durch das Gebiet $U$,
wird ein längeres Parameterintervall benötigt, um die Strecke
zwischen zwei Punkten zu durchlaufen.
Das gesuchte Wegintegral benötigt also auch noch einen
von der Geschwindigkeit $\dot{\gamma}(t)$ abhängigen Faktor,
um diesen Unterschied auszugleichen.

\begin{definition}[Wegintegral in $\mathbb{C}$]
Sei $f\colon U\to\mathbb{C}$ eine stetige komplexe Funktion und
$\gamma:I\to U$ ein differenzierbarer Weg $U$ mit $I=[a,b]$.
Dann ist das \emph{Integral von $f$ entlang des Weges $\gamma$}
gegeben durch
\begin{equation}
\int_\gamma f(z)\,dz
=
\int_I f(\gamma(t)) \dot{\gamma}(t)\,dt.
\end{equation}
Die gleiche Definition kann verwendet werden, wenn der Weg
$\gamma$ in endlich vielen Punkten nicht differenzierbar ist.
Wenn $\gamma$ ein geschlossener Weg ist, wird das Integral auch
\[
\int_\gamma f(z)\,dz
=
\oint_\gamma f(z)\,dz
\]
geschrieben.
\end{definition}

Wir müssen überprüfen, dass diese Definition tatsächlich nicht
von der Parametrisierung abhängt.
Dazu sei $t(\tau)$ eine invertierbare Funktion, die auf einem Intervall
$I_1=[c,d]$ definiert ist.
Nach der Formel für den Koordinatenwechsel in einem Integral
folgt dann
\begin{align*}
\int_c^d f(\gamma(t(\tau))) \frac{d\gamma(t(\tau))}{d\tau} \,d\tau
&=
\int_c^d f(\gamma(t(\tau))) \dot{\gamma}(t(\tau)) \frac{dt(\tau)}{dt}\,d\tau
\\
&=
\int_{\tau(c)}^{\tau(d)} f(\gamma(t)) \dot{\gamma}(t) \,dt
\\
&=
\int_a^b f(\gamma(t)) \dot{\gamma}(t) \,dt.
\end{align*}
Die Umparametrisierung $\tau(t)$ ändert also den Wert des Integrals
nicht.

\begin{beispiel}
Wir berechnen das Wegintegral für den geschlossenen Weg von
Beispiel~\ref{buch:integration:wege:bsp:kreis}
für die Funktion $f(z) = z^n$.
Einsetzen der Parametrisierung ergibt
\begin{align*}
\oint_\gamma f(z)\,dz
&=
\int_0^{2\pi}
\bigl( e^{it} \bigr)^{n}\,d ie^{it}
\,dt
=
i
\int_0^{2\pi}
e^{i(n+1)t}
\,dt
=
i
\int_0^{2\pi} \cos (n+1)t + i\sin(n+1)t\,dt.
\end{align*}
Falls $n+1\ne 0$ ist, stehen auf der rechten Seite Integrale der
Sinus- und Kosinusfunktion über ein ganze Anzahl von Perioden, diese
Integrale verschwinden.
Für $n=-1$ jedoch verschwindet der Sinus-Term und es bleibt
\[
\oint_\gamma \frac{dz}z
=
i
\int_0^{2\pi}
1\,dt
=
2\pi i.
\]
Zusammengefasst erhalten wir für ganzzahliges $n$
\[
\oint_\gamma z^n\,dtz
=
\begin{cases}
0&\qquad\text{falls $n\ne -1$}\\
2\pi i&\qquad\text{falls $n=-1$}
\end{cases}
\qedhere
\]
\end{beispiel}

Wir schreiben $\gamma(t) = x(t) + iy(t)$ und die Funktion $f=u+iv$ 
als Summe von Real- und Imaginärteil.
Dann wird das Wegintegral zu
\begin{align*}
\int_\gamma f(z)\,dz
&=
\int_a^b \bigl(u(x(t), y(t)) + i v(x(t), y(t))\bigr)\,
(\dot{x}(t)+ i\dot{y}(t))\,dt
\\
&=
\int_a^b u(x(t),y(t)) \dot{x}(t) -  v(x(t),y(t))\dot{y}(t)\,dt
\\
&\qquad
+
i
\int_a^b u(x(t),y(t))\dot{y}(t) + v(x(t),y(t))\dot{x}(t)\,dt
\end{align*}

%
% Wegintegral und Stammfunktion
%
\subsubsection{Wegintegral und Stammfunktion}

%
% Homotopie
%
\subsection{Homotopie}

