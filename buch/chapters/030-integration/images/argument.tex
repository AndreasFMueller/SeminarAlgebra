%
% argument.tex -- Argumentfunktion für den Pfad
%
% (c) 2021 Prof Dr Andreas Müller, OST Ostschweizer Fachhochschule
%
\documentclass[tikz]{standalone}
\usepackage{amsmath}
\usepackage{times}
\usepackage{txfonts}
\usepackage{pgfplots}
\usepackage{csvsimple}
\usetikzlibrary{arrows,intersections,math}
\begin{document}
\definecolor{darkred}{rgb}{0.8,0,0}
\definecolor{darkgreen}{rgb}{0,0.6,0}
\definecolor{gelb}{rgb}{1,0.8,0}
\def\skala{1}
\def\w{70}
\begin{tikzpicture}[>=latex,thick,scale=\skala,
declare function = {
winkel(\t) = 360*(5*\t+cos(360*\t)+0.3*sin(3*360*\t))-50;
r(\t) = 1.5 + 0.0*(sin(360*\t+90));
}]

\begin{scope}[xshift=-3.4cm]
\fill[color=blue!10] (0,0) -- (0:2) arc (0:\w:2);
\draw[color=blue] (0,0) -- (\w:3.1);
\node[color=blue] at (\w:3.1) [right] {$s$};
\node[color=blue] at ({0.5*\w}:1.6) {$\varphi_0$};
\draw[color=darkred] plot[domain=0:1,samples=1000]
	({winkel(\x)}:{1.6+1.2*cos(360*\x+90)})
	-- cycle;
\draw[color=darkgreen] plot[domain=0:1,samples=1000]
	({360*(5*\x+1)-50}:{1.6+1.2*cos(360*\x+90)})
	-- cycle;
\draw[line width=0.2pt] (-3,0) -- (3,0);
\draw[line width=0.2pt] (0,-3) -- (0,3);
\draw[->] (-3,-2) -- (3,-2) coordinate[label={below:$\operatorname{Re}$}];
\draw[->] (-1,-3) -- (-1,3) coordinate[label={right:$\operatorname{Im}$}];
\node at (0,0) [below left] {$z_0$};
\fill[color=white] (0,0) circle[radius=0.05];
\draw (0,0) circle[radius=0.05];
\node at (-3,3) [below right] {$\mathbb{C}$};
\end{scope}

\begin{scope}[yshift=-2.9cm,xshift=0.0cm]
\fill[color=blue!10] (0,{(360-\w)/360}) rectangle (5,1);
\fill[color=gelb] (0.16,1) circle[radius=0.1];
\fill[color=gelb] (0.81,1) circle[radius=0.1];
\fill[color=gelb] (1.82,1) circle[radius=0.1];
\fill[color=gelb] (3.17,2) circle[radius=0.1];
\fill[color=gelb] (3.48,3) circle[radius=0.1];
\fill[color=gelb] (3.89,4) circle[radius=0.1];
\fill[color=gelb] (4.68,5) circle[radius=0.1];

\node[color=brown] at (0.16,1) [above left] {$\scriptstyle +1$};
\node[color=brown] at (0.81,1) [above right] {$\scriptstyle -1$};
\node[color=brown] at (1.82,1) [above left] {$\scriptstyle +1$};
\node[color=brown] at (3.17,2) [below right] {$\scriptstyle +1$};
\node[color=brown] at (3.48,3) [below right] {$\scriptstyle +1$};
\node[color=brown] at (3.89,4) [below right] {$\scriptstyle +1$};
\node[color=brown] at (4.68,5) [below right] {$\scriptstyle +1$};

\draw[line width=0.2pt] (5,0) -- (5,6);
\foreach \y in {0,...,5}{
	\draw[color=blue,line width=0.2pt] (0,\y) -- (5,\y);
}
\draw[color=darkgreen] (0,{(winkel(0)-\w)/360)}) -- (5,{(winkel(1)-\w)/360});
\draw[color=darkred] plot[domain=0:1,samples=100]
	({5*\x},{(winkel(\x)-\w)/360});
\draw[->] (-0.1,{(360-\w)/360}) -- ++(6.0,0) coordinate[label={below:$t$}];
\draw[->] (0,-0.1) -- (0,6) coordinate[label={right:$\varphi$}];
\draw[color=blue] (0,1) -- ++(5,0);
\node[color=blue] at (5,1) [right] {$\varphi_0$};
\node[color=blue] at (5,2) [right] {$+2\pi$};
\node[color=blue] at (5,3) [right] {$+4\pi$};
\node[color=blue] at (5,4) [right] {$+6\pi$};
\node[color=blue] at (5,5) [right] {$+8\pi$};
\node[color=blue] at (5,0) [right] {$-2\pi$};
\end{scope}

\end{tikzpicture}
\end{document}

