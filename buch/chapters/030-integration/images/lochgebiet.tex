%
% lochgebiet.tex -- Gebiet mit Loch
%
% (c) 2021 Prof Dr Andreas Müller, OST Ostschweizer Fachhochschule
%
\documentclass[tikz]{standalone}
\usepackage{amsmath}
\usepackage{times}
\usepackage{txfonts}
\usepackage{pgfplots}
\usepackage{csvsimple}
\usetikzlibrary{arrows,intersections,math}
\begin{document}
\def\skala{1}
\definecolor{darkred}{rgb}{0.8,0,0}
\begin{tikzpicture}[>=latex,thick,scale=\skala]

\coordinate (A) at (1,2);
\coordinate (B) at (7,2);
\coordinate (C) at (3,6);
\coordinate (D) at (5,3.5);

\coordinate (z0) at (2,3);

\fill[color=gray!40]
	(A) to[out=-30,in=-150]
	(B) to[out=30,in=10]
	(C) to[out=-170,in=150]
	cycle;

\fill[color=white]
	(D) circle[radius=0.5];

\fill[color=white] (z0) circle[radius=0.05];
\draw (z0) circle[radius=0.05];
\node at (z0) [right] {$z_0$};

\draw[->] (-0.1,0) -- (8,0) coordinate[label={$\operatorname{Re}$}];
\draw[->] (0,-0.1) -- (0,6.5) coordinate[label={right:$\operatorname{Im}$}];
\node at (8,6.5) [below left] {$\mathbb{C}$};

\draw[color=darkred] (4,2.5) [rotate=10] ellipse(2.4cm and 1cm);
\node[color=darkred] at (4,1.9) {$\gamma_1$};
\draw[color=blue] (4,0) [rotate=57] ellipse(1.5cm and 1cm);
\node[color=blue] at (2.7,5) {$\gamma_2$};

\node at (3.4,5.6) {$U$};

\end{tikzpicture}
\end{document}

