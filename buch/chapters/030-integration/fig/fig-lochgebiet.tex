%
% fig-lochgebiet.tex
%
% (c) 2026 Prof Dr Andreas Müller
%
\begin{figure}
\centering
\includegraphics{chapters/030-integration/images/lochgebiet.pdf}
\caption{Zwei Wege um den Punkte $z_0$ im Gebiet $U$.
Der Weg $\gamma_1$ führt um das Loch im Gebiet $U$ herum und lässt
sich daher nicht in den Punkt $z_0$ zusammenziehen.
Dies Art von Weg kann im Satz~\ref{buch:integration:fig:residuensatzn}
nicht verwendet werden.
Nur Wege wie $\gamma_2$ sind zulässig.
\label{buch:integration:fig:lochgebiet}}
\end{figure}
