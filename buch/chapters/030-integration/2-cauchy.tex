%
% 2-cauchy.tex -- 2. Cauchy-Integralsatz
%
% (c) 2025 Prof Dr Andreas Müller
%
\section{Der Residuensatz
\label{buch:integration:section:cauchy}}
\kopfrechts{Der Integralsatz von Cauchy}
Das Beispiel~\ref{buch:integration:wege:bsp:kreisintegral} hat
gezeigt, dass mit Ausnahme des Falls $n=-1$ die Potenzen $z^n$ 
das Wegintegral über einen Kreis um den Nullpunkt den Wert 0
ergab.
Für $\frac{1}{z}$ ergab sich der Wert
\[
\oint_{S^1} \frac{dz}{z}
=
2\pi i.
\]
Der Homotopiebegriff erlaubt, die Wege beliebig zu deformieren,
solange der Definitionsbereich der Funktion nicht verlassen wird.
Es scheint, dass das Integral über geschlossene Wege vor allem davon
abhängt, was in Polstellen des Integranden passiert.
Daher
Der Residuensatz, der in diesem Abschnitt hergeleitet werden soll,
formalisiert diese Beobachtung.

%
% Wegintegrale über geschlossene Wege
%
\subsection{Wegintegrale über geschlossene Wege}

%
% Wegintegrale und Pole
%
\subsection{Wegintegrale und Pole}
Sei $f\colon U\to\mathbb{C}$ eine holomorphe Funktion und
$a\in U$ ein Punkt.
Sei weiter $\gamma$ ein geschlossener Weg, der den Punkt $a$
einmal im Gegenuhrzeigersinn umkreist.

%
% Mittelwerteigenschaft
%
\subsubsection{Mittelwerteigenschaft}
Aus der Formel~\eqref{XXX} folgt auch sofort die folgende Eigenschaft
einer harmonischen Funktion.

\begin{satz}[Mittelwerteigenschaft]
\label{buch:integration:cauchy:satz:mittelwerteigenschaft}
Sei $U\subset\mathbb{C}$ ein Gebiet und $z_0\in U$ ein Punkt in $U$.
Weiter sei $u$ eine harmonische Funktion in $U$ und $r$ der Radius
eines Kreises um den Punkt $z_0$, der immer noch ganz in $U$ enhalten ist.
Dann ist
\begin{equation}
u(z_0)
=
\frac{1}{2\pi}
\int_0^{2\pi} u(z_0 + re^{it}) \,dt.
\label{buch:integration:cauchy:eqn:mittelwerteigenschaft}
\end{equation}
\end{satz}

\begin{proof}
Zunächst kann nach Satz~\ref{buch:holomorph:holomorph:satz:utov}
eine Funktion $v$ gefunden werden, so dass in einer Umgebung, die auch
den Kreis mit Radius $r$ um $z_0$ enthält, $f(z) = u(z) + iv(z)$ eine
holomorphe Funktion ist.
Nach dem Residuensatz ist dann
\[
f(z_0)
=
\frac{2\pi i}
\oint_{S^1_r} 
\frac{f(z)}{z-z_0}
\,dz
\]
für den Kreis $S^1_r$ mit Radius $r$ um den Punkt $z_0$.
Das Integral kann man durch $t\mapsto z_0+e^{it}$ parametrisieren,
dann wird es zu
\begin{align*}
f(z_0)
&=
\frac{1}{2\pi i}
\int_0^{2\pi}
\frac{ f(z_0+re^{it}) }{z_0+re^{it}-z_0} ir e^{it}\,dt
=
\frac{1}{2\pi}
\int_0^{2\pi}
\frac{ f(z_0+re^{it}) }{re^{it}} r e^{it}\,dt
\\
&=
\frac{1}{2\pi}
\int_0^{2\pi}
 f(z_0+re^{it})\,dt.
\intertext{Den Integranden kann man in Real- und Imaginärteil aufspalten und
erhält}
u(z) + iv(z)
&=
\frac{1}{2\pi}
\int_0^{2\pi}
u(z_0+re^{it})\,dt
+
\frac{i}{2\pi}
\int_0^{2\pi}
v(z_0+re^{it})\,dt.
\end{align*}
Die Formel~\eqref{buch:integration:cauchy:eqn:mittelwerteigenschaft}
gilt daher sowohl für $u$ wie auch für $v$ ganz unabhängig von der
Wahl der Funktion $v$.
\end{proof}

%
% Maximumprinzip
%
\subsubsection{Maximumprinzip}
Aus der Mittelwerteigenschaft folgt eine weitere wichtige Eigenschaft
harmonischer Funktionen, nämlich das Maximum- und Minimum-Prinzip.

\begin{satz}[Maximumprinzip]
Sei $U\subset\mathbb{C}$ ein Gebiet und $u(x,y)$ ein in $U$ harmonische
Funktion.
Dann nimmt die Funktio $u$ ihr Maximum und Minimum auf dem Rand von $U$ an.
\end{satz}

\begin{proof}
Wir zeigen, dass das Maximum oder Minimum nicht im Inneren des Gebietes
$U$ liegen kann.
Dazu nehmen wir an, das Maximum liegt im Punkt $z_0\in U$.
Sei $r>0$ der Radius eines Kreises um den Punkt $z_0$, der vollständig
in $U$ enthalten ist.
Dann ist $u(z_0+ re^{i\varphi}) < u(z_0)$ für alle Winkel $\varphi$,
die Werte auf dem Kreis sind kleiner als der Wert im Punkt $z_0$.
Da die Werte auf dem Kreis kleiner sind als der Wert $u(z_0)$,
ist der Mittelwert $<u(z_0)$.
Nach Satz~\ref{buch:integration:cauchy:satz:mittelwerteigenschaft} ist
der Mittelwert der Werte auf dem Kreis mit Radius $r$ um $z_0$ gleich dem
Wert $u(z_0)$, ein Widerspruch.
Der Widerspruch zeigt, dass $z_0$ nicht ein Maximum sein kann.

Auf die gleiche Weise kann man den Fall eines Minimums behandeln.
\end{proof}

Das Maximumprinzip gilt für die Lösungen einer homogenen partiellen
Differentialgleichung mit einem beliebigen elliptischen Operator.
Der Beweis folgt allerdings einem ganz anderen Weg, denn die
Mittelwerteigenschaft gilt für solche Lösungen nicht.
Für eine ausführliche Diskussion siehe \cite[section 6.4]{buch:evans}.

%
% Ableitungen
%
\subsection{Ableitungen}

%
% Die Umlaufzahl
%
\subsection{Die Umlaufzahl}





