%
% 2-cauchy.tex -- 2. Cauchy-Integralsatz
%
% (c) 2025 Prof Dr Andreas Müller
%
\section{Der Residuensatz
\label{buch:integration:section:cauchy}}
\kopfrechts{Der Integralsatz von Cauchy}
Das Beispiel~\ref{buch:integration:wege:bsp:kreisintegral} hat
gezeigt, dass mit Ausnahme des Falls $n=-1$ die Potenzen $z^n$ 
das Wegintegral über einen Kreis um den Nullpunkt den Wert 0
ergab.
Für $\frac{1}{z}$ ergab sich der Wert
\[
\oint_{S^1} \frac{dz}{z}
=
2\pi i.
\]
Der Homotopiebegriff erlaubt, die Wege beliebig zu deformieren,
solange der Definitionsbereich der Funktion nicht verlassen wird.
Es scheint, dass das Integral über geschlossene Wege vor allem davon
abhängt, was in Polstellen des Integranden passiert.
Daher
Der Residuensatz, der in diesem Abschnitt hergeleitet werden soll,
formalisiert diese Beobachtung.

%
% Wegintegrale über geschlossene Wege
%
\subsection{Wegintegrale über geschlossene Wege}

%
% Wegintegrale und Pole
%
\subsection{Wegintegrale und Pole}

%
% Ableitungen
%
\subsection{Ableitungen}

%
% Die Umlaufzahl
%
\subsection{Die Umlaufzahl}





